\chapter[Preface]{Preface\LARGE{\footnotemark{}}\footnotetext{The historical
information is taken from \smartcite{churchandcurry} and
\smartcite{lambdaimpact} for lambda calculus, and \smartcite{troelstrahist} and
\smartcite{basicprooftheory} for intuitionistic mathematics.}}

The two names usually connected with lambda calculus are those of Alonzo Church
and Haskell Curry. While both their work was groundlaying for the different
well-known systems of lambda calculus in use today, they originally set out to
design type-free formal systems that would allow capturing the foundations of
mathematics in a logical language with functional instead of set theoretic
notions. Both independently developed systems that were much larger than what is
known today as lambda calculus, or, in the case of Curry, combinatory logic.

It should turn out though, as was shown by Church's students Stephen Kleene and
John Rosser in 1934, that both systems were inconsistent as they allowed the
derivation of Richard's paradox. While this led Curry to attempt to correct his
approach, Church gave up on the foundational project and cocentrated his efforts
on those consistent parts of his system that had shown fruitful already in 1933
for making exact the notion of intuitively computable functions: the theory of
lambda terms. Along with the Turing Machine, this revelation should prove to be
foundational for the emerging field of computer science.

At around the same time in Europe, Andrei Kolmogorov from as early as 1925 and
Arendt Heyting from 1930 independently published articles in the seemingly
unrelated field of intuitionistic mathematics. In their work, they gave formal
systems for logic as it was interpreted in intuitionism, thereby making explicit
the interpretation of the logical connectives that is mainly due to Brouwer.
Brouwer saw as mathematical objects those entities that could be constructed in
the mind of the mathematician. This is in contrast to realist conceptions that
place mathematical objects outside of the mind, aswell as formalist conceptions
that cut ties of mathematics to any actual objects. While making mathematical
objects more malleable and accessible to the mathematician, intuitionism
precludes her from certain ways of reasoning about them. While for example the
realist has no trouble making a disjunctive statement about \textit{real}
objects, the intuitionist has nothing to make a disjunctive statement about if
she can not construct it in her mind, viz. show that at least one of the
disjuncts holds.

It should be clear that this conception of mathematics is incompatible with the
connectives of classical logic, as those would allow reasoning about
mathematical objects in the abstract which may lead to empty statements
possessing no referents. Instead, in order to make sensible propositions, the
mathematician needs to be able to construct what he is referring to. Naturally,
the notion of proofs is apt to change with the notions behind the connectives.
In this light, a step in a proof is valid if it is known how to constructively
prove the conclusion if its premises have been proven in the same manner. This
constructive perspective on proofs appears strikingly procedural and suggests
looking at proofs as objects that can be combined by rule application to form
new proofs. With some aquaintance with lambda calculus, and relatedly functional
programming, this understanding of proofs bears an uncanny resemblance to
different operations on functions. And indeed, in the years following the Second
World War, this connection has become more and more evident, being explicitly
shown for different proof systems and models of computation from the late 1950s.

The overall result is known as the \textit{Curry-Howard correspondence}. In this
report, we want to illuminate the correspondence between natural deduction for
propositional implicational logic and simply-typed lambda calculus. We will use
this result to look at parallels in normalisation of natural deduction proofs
and lambda terms, using the confluence (Church-Rosser) theorem to show the
consistency of \implnpi. Although not technically necessary, we will subscribe
to the \textit{Brouwer-Heyting-Kolmogorov interpretation} of proofs in
intuitionistic logic, because it provides helpful intuitions for the
correspondence with lambda calculus.
