\chapter{Curry-Howard Correspondence}

Parts of the formalisms introduced in the prerequisites appear to function
largely in parallel. When we look at the effects of functional abstraction and
function application on types in \stlambda, we might be reminded of implication
introduction and implication elimination in \implnpi. And indeed, this apparent
structural similarity can be grasped formally and turns out to be so strong that
the two formalisms are isomorphic. To support this claim we obviously need to
establish the existence of a bijection between the two systems. We can then show
that the aforementioned operations are indeed invariant under the bijection.

In the literature we referred to, proofs of the isomorphism were largely
suggestive. Although we neither want to claim to give a totally complete proof,
nor want give a totally complete proof, we were interested in seeing the details
work out and thus try to take a more formal route compared to the sources.

\section{Smooth Operators}

\begin{definition}
We define the notation $M^{A}$ for a simply-typed lambda term $M$ to mean that
$M$ contains the outermost type annotation $A$, i.e. $M^A \equiv M':A$ for some
string $M'$.
\end{definition}

\begin{theorem}[Curry-Howard Isomorphism]

The natural deduction system \implnpi\ consisting of the set $\npiproofs$ of
deduction trees with implication introduction and implication elimination as
operations is isomorphic to the term calculus \stlambda\ consisting of the set
$\stlambdaterms$ of simply-typed lambda terms with functional abstraction and
function application as operations.

\end{theorem}

\begin{proof}
We give a bijection $\chmap: \npiproofs \to \stlambdaterms$ from the set $\npiproofs$
of \implnpi\ deduction trees to the set $\stlambdaterms$ of simply-typed lambda terms.

\begin{alignat*}{2}
\chmap(A^u) &:= u: A \\[6pt]
\chmap\left(
  \AxiomC{$\mathcal{D}_1$}
  \noLine
  \UnaryInfC{$A \to B$}
  \AxiomC{$\mathcal{D}_2$}
  \noLine
  \UnaryInfC{$A$}
  \RightLabel{$\to$E}
  \BinaryInfC{$B$}
  \DisplayProof
\right) &:= \chmap(\mathcal{D}_1) \chmap(\mathcal{D}_2): B \\[6pt]
\chmap\left(
  \AxiomC{$[A]^u$}
  \noLine
  \UnaryInfC{$\mathcal{D}$}
  \noLine
  \UnaryInfC{$B$}
  \RightLabel{$\to$I, $u$}
  \UnaryInfC{$A \to B$}
  \DisplayProof
\right) &:= (\lambda u: A.\chmap(\mathcal{D})): A \to B
\end{alignat*}

\begin{proposition}
If $\mathcal{D}$ is a deduction of $A$, then $\chmap(\mathcal{D})$ has type $A$.
\end{proposition}

\begin{proof}
By induction on the complexity of $\mathcal{D}$.
\end{proof}

We additionally define a function $\stlambdaterms \to \npiproofs$, suggestively
named $\chmapanta$:

\begin{alignat*}{2}
\chmapanta(u: A) & := A^u \\[6pt]
\chmapanta((M^{A \to B})(N^A): B) &:=
  \AxiomC{$\chmapanta(M^{A \to B})$}
  \noLine
  \UnaryInfC{$A \to B$}
  \AxiomC{$\chmapanta(N^A)$}
  \noLine
  \UnaryInfC{$A$}
  \RightLabel{$\to$E}
  \BinaryInfC{B}
  \DisplayProof \\[6pt]
\chmapanta((\lambda u:A.M^B):A \to B) &:=
  \AxiomC{$\chmapanta(M^B)$}
  \noLine
  \UnaryInfC{$B$}
  \RightLabel{$\to$I, $u$}
  \UnaryInfC{$A \to B$}
  \DisplayProof
\end{alignat*}

\begin{proposition}
For any lambda term $M^A$, $\chmapanta(M^A)$ is a proof of $A$.
\end{proposition}

\begin{proof}
By induction on the complexity of lambda terms.
\end{proof}

We now show that $\chmap$ is a bijection by showing that
\begin{itemize}
\item[(i)] $\chmapanta \circ \chmap = id_{\npiproofs}$ and
\item[(ii)] $\chmap \circ \chmapanta = id_{\stlambdaterms}$.
\end{itemize}
The proof is by induction on the construction of deduction trees and lambda
terms respectively.

\[
\chmapanta(\chmap(A^u)) = \chmapanta(u:A) = A^u = id_{\npiproofs}(A^u)
\]

\begin{align*}
\chmapanta\left(\chmap\left(
  \AxiomC{$\mathcal{D}_1$}
  \noLine
  \UnaryInfC{$A \to B$}
  \AxiomC{$\mathcal{D}_2$}
  \noLine
  \UnaryInfC{$A$}
  \RightLabel{$\to$E}
  \BinaryInfC{$B$}
  \DisplayProof
\right)\right)
& =
  \chmapanta(\chmap(\mathcal{D}_1) \chmap(\mathcal{D}_2): B)
  \tag{By def. $\chmap$} \\[6pt]
& =
  \AxiomC{$\chmapanta(\chmap(\mathcal{D}_1))$}
  \noLine
  \UnaryInfC{$A \to B$}
  \AxiomC{$\chmapanta(\chmap(\mathcal{D}_2))$}
  \noLine
  \UnaryInfC{$A$}
  \RightLabel{$\to$E}
  \BinaryInfC{B}
  \DisplayProof \tag{By def. $\chmapanta$} \\[6pt]
& =
  \AxiomC{$\mathcal{D}_1$}
  \noLine
  \UnaryInfC{$A \to B$}
  \AxiomC{$\mathcal{D}_2$}
  \noLine
  \UnaryInfC{$A$}
  \RightLabel{$\to$E}
  \BinaryInfC{$B$}
  \DisplayProof \tag{By induction hypothesis} \\[6pt]
& =
  id_{\npiproofs}\left(
  \AxiomC{$\mathcal{D}_1$}
  \noLine
  \UnaryInfC{$A \to B$}
  \AxiomC{$\mathcal{D}_2$}
  \noLine
  \UnaryInfC{$A$}
  \RightLabel{$\to$E}
  \BinaryInfC{$B$}
  \DisplayProof
  \right)
\end{align*}

\begin{align*}
\chmapanta\left(\chmap\left(
  \AxiomC{$[A]^u$}
  \noLine
  \UnaryInfC{$\mathcal{D}$}
  \noLine
  \UnaryInfC{$B$}
  \RightLabel{$\to$I, $u$}
  \UnaryInfC{$A \to B$}
  \DisplayProof
\right)\right)
& =
  \chmapanta((\lambda u: A.\chmap(\mathcal{D})): A \to B)
  \tag{By def. $\chmap$} \\[6pt]
& =
  \AxiomC{$\chmapanta(\chmap(\mathcal{D}))$}
  \noLine
  \UnaryInfC{$B$}
  \RightLabel{$\to$I, $u$}
  \UnaryInfC{$A \to B$}
  \DisplayProof \tag{By def. $\chmapanta$} \\[6pt]
& =
  \AxiomC{$[A]^u$}
  \noLine
  \UnaryInfC{$\mathcal{D}$}
  \noLine
  \UnaryInfC{$B$}
  \RightLabel{$\to$I, $u$}
  \UnaryInfC{$A \to B$}
  \DisplayProof \tag{By induction hypothesis} \\[6pt]
& =
  id_{\npiproofs}\left(
  \AxiomC{$[A]^u$}
  \noLine
  \UnaryInfC{$\mathcal{D}$}
  \noLine
  \UnaryInfC{$B$}
  \RightLabel{$\to$I, $u$}
  \UnaryInfC{$A \to B$}
  \DisplayProof
  \right)
\end{align*}

\begin{align*}
\chmap(\chmapanta(u:A)) & = \chmap(A^u) = u:A = id_{\stlambdaterms}(u:A)
\end{align*}

\begin{align*}
\chmap(\chmapanta(M^{A \to B}N^A:B)) & =
  \chmap\left(
  \AxiomC{$\chmapanta(M^{A \to B})$}
  \noLine
  \UnaryInfC{$A \to B$}
  \AxiomC{$\chmapanta(N^A)$}
  \noLine
  \UnaryInfC{$A$}
  \RightLabel{$\to$E}
  \BinaryInfC{B}
  \DisplayProof
  \right) \tag{By def. $\chmapanta$}\\[4pt]
& =
  \chmap(\chmapanta(M^{A \to B}))\chmap(\chmapanta(N^A)): B
  \tag{By def.  $\chmap$}\\[6pt]
& =
  M^{A \to B}N^A:B
  \tag{By induction hypothesis}\\[6pt]
& =
  id_{\stlambdaterms}(M^{A \to B}N^A:B)
\end{align*}

\begin{align*}
\chmap(\chmapanta((\lambda u:A.M^B):A \to B)) &=
  \chmap\left(
  \AxiomC{$\chmapanta(M^B)$}
  \noLine
  \UnaryInfC{$B$}
  \RightLabel{$\to$I, $u$}
  \UnaryInfC{$A \to B$}
  \DisplayProof
  \right) \tag{By def. $\chmapanta$} \\[6pt]
& =
  (\lambda u: A.\chmap(\chmapanta(M^B))): A \to B
  \tag{By def. $\chmap$} \\[6pt]
& =
  (\lambda u: A.M^B): A \to B
  \tag{By induction hypothesis} \\[6pt]
& =
  id_{\stlambdaterms}((\lambda u: A.M^B): A \to B)
\end{align*}

The correspondence of implication introduction with functional abstraction, and
implication elimination with function application is immediate from the
definition of $\chmap$ and $\chmapanta$.
\end{proof}

\section{Combinators and Axioms}
\label{section:ski}

The main result of the previous section is known under various names, among them
the \textit{formulas-as-types} and \textit{proofs-as-programs} interpretations.
When we showed three exemplary deductions of axioms of propositional logic in
\ref{sec:naturaldeduction}, we took care to sneak in something that would make
sense in the light of these interpretations. The three axioms that were deduced
correspond to typed instances of three \textit{combinators} that are known as
$\mathsf{I}$, $\mathsf{K}$ and $\mathsf{S}$ in the literature:
\begin{align*}
\mathsf{I} &=
  (\lambda u^A.u)^A \\
\mathsf{K} &=
  (\lambda u^A.
     (\lambda v^B.u)
  )^{A \to B \to A} \\
\mathsf{S} &=
  (\lambda u^{A \to B \to C}.
     (\lambda w^{A \to B}.
       (\lambda v^A.(uv)(wv))
     )
  )^{(A \to B \to C) \to ((A \to B) \to (A \to C))}
\end{align*}
The system consisting of the $\mathsf{SKI}$ combinators is a well-studied Turing
complete system. While $\mathsf{SKI}$ due to its bare manner is of theoretical
interest only, it is not a minimal system. Instead, already the system
$\mathsf{SK}$ is Turing complete, as $\mathsf{I}$ can be obtained by combining
$\mathsf{S}$ and $\mathsf{K}$. Looking at the combinators as given above, we see
a problem with this. We can neither apply $\mathsf{S}$ to $\mathsf{K}$ nor
$\mathsf{K}$ to $\mathsf{S}$, because the types don't match.

Unless we treat each of the combinators as a schema to be instantiated by
assigning types to the type meta-variables occuring in the expression, we will
have to construct appropriate combinators for specific types. Because of the
workings of \implnpi\ we know that we could always do that by using the
appropriate simple types from $\Pi$ as assumptions in the appropriate places. We
can therefore justify the meta-rule of using the combinators above as schemas.

We will now combine appropriate instances of $\mathsf{S}$ and $\mathsf{K}$ to
yield $\mathsf{I}$. To better manage the size of the expressions, we will omit
types where unproblematic and use variables for terms.
\begin{align*}
\mathsf{S} &=
(\lambda u^{(A \to (B \to A) \to A) \to (A \to (B \to A)) \to (A \to A)}.
  (\lambda w^{A \to B \to A}.
    (\lambda v^A.(uv)(wv))
  )
) \\
\mathsf{K_1} &=
(\lambda u_1^A.
  (\lambda v_1^{B \to A}.u)
) \\
\mathsf{K_2} &=
(\lambda u_2^A.
  (\lambda v_2^B.u)
) \\
\mathsf{(SK_1)K_2} &=_\beta
(
  (\lambda u^{(A \to (B \to A) \to A) \to (A \to (B \to A)) \to (A \to A)}.
    (\lambda w^{A \to B \to A}.
      (\lambda v^A.(uv)(wv))
    )
  )
  \mathsf{K_1}
)
\mathsf{K_2} \\
&=_\beta
(\lambda w^{A \to B \to A}.
  (\lambda v^A.(\mathsf{K_1}v)(wv))
)
\mathsf{K_2} \\
&=_\beta
(\lambda v^A.(\mathsf{K_1}v)(\mathsf{K_2}v)) \\
&=_\beta
(\lambda v^A.
  ((\lambda u_1^A.
    (\lambda v_1^{B \to A}.u)
  )v)
  (\mathsf{K_2}v)
) \\
&=_\beta
(\lambda v^A.
  (\lambda v_1^{B \to A}.v)
  (\mathsf{K_2}v)
) \\
&=_\beta
(\lambda v^A.
  (\lambda v_1^{B \to A}.v)
  ((\lambda u_2^A.
     (\lambda v_2^B.u_2)
  )v)
) \\
&=_\beta
(\lambda v^A.
  (\lambda v_1^{B \to A}.v)
  (\lambda v_2^B.v)
) \\
&=_\beta
(\lambda v^A.v) \\
&=
\mathsf{I}
\end{align*}
We were thus able to derive $\mathsf{I}$, given we identify terms that differ only in
the names of bound variables.

It would be interesting to see what transformations are being made on the
corresponding deduction trees while following steps of $\beta$-reduction.  Alas,
the terms above become very complicated when translated into deductions. We will
have to look at a simpler example.

\section{Transformations}

Since by \ref{cor:uniquenessoftypes} we know that the types of any terms, that
are equivalent under $\to_\beta$, coincide, we can say that the deductions
corresponding to $(\mathsf{SK_1})\mathsf{K_2}$ and $\mathsf{I}$ must possess the
same conclusion. Clearly, the deduction corresponding to $\mathsf{I}$ will be
more economical. We can also see that the term $\mathsf{I}$ is in normal form.
This example could tempt us to think that the normal form is a way of
economically notating a function or that proofs corresponding to normal forms
are more concise. But this is not generally the case.\\
\\
Consider the term
\[
(\lambda v^{A \to B}.
  v(u^{(A \to B) \to A}v)
)(
  w^{(C \to D) \to (A \to B)}(\lambda y^C.x^D)
)
\]
and its one-step conversum
\[
(w^{(C \to D) \to (A \to B)}(\lambda y^C.x^D))(u^{(A \to B) \to A}w^{(C \to D) \to (A \to B)}(\lambda y^C.x^D))
\]
with corresponding deduction trees
\begin{prooftree}
\scriptsize
\implelim{%
  \implintro{%
    \implelim{\AxiomC{$A \to B^v$}}{%
      \implelim{\AxiomC{$(A \to B) \to A^u$}}{\AxiomC{$A \to B^v$}}{A}
    }{$B$}
  }{v}{$(A \to B) \to B$}
}{%
  \implelim{\AxiomC{$(C \to D) \to (A \to B)^w$}}{%
    \implintro{\AxiomC{$D^x$}}{y}{$C \to D$}
  }{$A \to B$}
}{$B$}
\end{prooftree}
and
\begin{prooftree}
\scriptsize
\implelim{%
  \implelim{\AxiomC{$(C \to D) \to (A \to B)^w$}}{%
    \implintro{\AxiomC{$D^x$}}{y}{$C \to D$}
  }{$A \to B$}
}{%
  \implelim{%
    \AxiomC{$(A \to B) \to A^u$}
  }{%
    \implelim{\AxiomC{$(C \to D) \to (A \to B)^w$}}{%
      \implintro{\AxiomC{$D^x$}}{y}{$C \to D$}
    }{$A \to B$}
  }{A}
}{$B$}
\end{prooftree}
. Both do not seem like improvements in use of space, ink or memory. And looking
at it from the perspective of lambda functions this makes intuitive sense:
Functional abstraction allows us to assign a name to a term and then refer to it
multiple times without writing out the full corresponding expression each time.
Analogously, the use of detours in deductions allows us to appeal to
another proof multiple times without spelling it out over and over. Then, if
necessary, we can instantiate the proof at the appropriate positions. In hope to
transfer more ideas from one structure to the other, we will look at the
relation of normalisation in deductions and lambda terms in some detail.

\section{Normalisation}

\begin{definition}[Proof Instantiation]
We define a function for replacing all occurences of an open assumption $A^u$ in
a deduction tree $\mathcal{D}$ by a deduction tree $\mathcal{E}$, written as
$\mathcal{D}[A^u/\mathcal{E}]$.

\[
  \mathcal{D}[A^u/\mathcal{E}] =
  \left\{
  \begin{array}{ll}
  \mathcal{E} & \mbox{if } \mathcal{D} = A^u \\[6pt]
  \mathcal{D} & \mbox{if } \mathcal{D} = B^v, v \neq u \\[10pt]
  \mathcal{D} & \mbox{if } \mathcal{D} =
    \AxiomC{$\mathcal{D}'$}
    \noLine
    \UnaryInfC{$C$}
    \RightLabel{$\to$I, $u$}
    \UnaryInfC{$B \to C$}
    \DisplayProof \\[20pt]
  \AxiomC{$\mathcal{D}'[A^u/\mathcal{E}]$}
  \noLine
  \UnaryInfC{$C$}
  \RightLabel{$\to$I, $v$}
  \UnaryInfC{$B \to C$}
  \DisplayProof & \mbox{if } \mathcal{D} =
    \AxiomC{$\mathcal{D}'$}
    \noLine
    \UnaryInfC{$C$}
    \RightLabel{$\to$I, $v$}
    \UnaryInfC{$B \to C$}
    \DisplayProof, v \neq u \\[24pt]
  \AxiomC{$\mathcal{D}'[A^u/\mathcal{E}]$}
  \noLine
  \UnaryInfC{$B \to C$}
  \AxiomC{$\mathcal{D}''[A^u/\mathcal{E}]$}
  \noLine
  \UnaryInfC{$B$}
  \RightLabel{$\to$E}
  \BinaryInfC{$C$}
  \DisplayProof & \mbox{if } \mathcal{D} =
    \AxiomC{$\mathcal{D}'$}
    \noLine
    \UnaryInfC{$B \to C$}
    \AxiomC{$\mathcal{D}''$}
    \noLine
    \UnaryInfC{$B$}
    \RightLabel{$\to$E}
    \BinaryInfC{$C$}
    \DisplayProof
  \end{array}
  \right.
\]
\end{definition}

\begin{definition}[Conversion]
We define a relation $\leadsto \subseteq \npiproofs\times\npiproofs$ that captures the notion of
normalisation steps for \implnpi\ deductions.

\[
  \AxiomC{$[A]^u$}
  \noLine
  \UnaryInfC{$\mathcal{D}$}
  \noLine
  \UnaryInfC{$B$}
  \RightLabel{$\to$I, $u$}
  \UnaryInfC{$A \to B$}
  \AxiomC{$\mathcal{D}_1$}
  \noLine
  \UnaryInfC{$A$}
  \RightLabel{$\to$E}
  \BinaryInfC{$B$}
  \DisplayProof
\ \leadsto\ \mathcal{D}[A^u/\mathcal{D}_1]
\]

\[
  \AxiomC{$[A]^u$}
  \noLine
  \UnaryInfC{$\mathcal{D}$}
  \noLine
  \UnaryInfC{$B$}
  \RightLabel{$\to$I, $u$}
  \UnaryInfC{$A \to B$}
  \DisplayProof
\ \leadsto\ 
  \AxiomC{$[A]^u$}
  \noLine
  \UnaryInfC{$\mathcal{E}$}
  \noLine
  \UnaryInfC{$B$}
  \RightLabel{$\to$I, $u$}
  \UnaryInfC{$A \to B$}
  \DisplayProof
\mbox{if } \mathcal{D} \leadsto \mathcal{E}
\]

\[
  \AxiomC{$\mathcal{D}_1$}
  \noLine
  \UnaryInfC{$A \to B$}
  \AxiomC{$\mathcal{D}_2$}
  \noLine
  \UnaryInfC{$A$}
  \RightLabel{$\to$E}
  \BinaryInfC{$B$}
  \DisplayProof
\ \leadsto\ 
  \AxiomC{$\mathcal{E}_1$}
  \noLine
  \UnaryInfC{$A \to B$}
  \AxiomC{$\mathcal{E}_2$}
  \noLine
  \UnaryInfC{$A$}
  \RightLabel{$\to$E}
  \BinaryInfC{$B$}
  \DisplayProof
\mbox{ if } \mathcal{D}_1 \leadsto \mathcal{E}_1
\mbox{ or } \mathcal{D}_1 \leadsto \mathcal{E}_1
\]
\end{definition}

