\chapter{Conclusion}

We have seen that the structural similarities between the logical and
computational systems we examined are vast and allow a transfer of intuitions
among different domains. Although the fathers of the BHK interpretation will not
have had computers in mind, the concept of constructing the referents of one's
reasoning is a fruitful metaphor for what happens in a computer's memory during
the execution of a program.

A short overview of which concept from our term language corresponds to which
concept of our proof system may be given as follows\footnote{adapted from
\cite[67]{lectures}}:
\begin{align*}
\mbox{typed variable} &\leftrightarrow \mbox{assumption} \\
\mbox{term} &\leftrightarrow \mbox{deduction} \\
\mbox{type} &\leftrightarrow \mbox{formula} \\
\mbox{redex} &\leftrightarrow \mbox{deduction with detour} \\
\mbox{reduction} &\leftrightarrow \mbox{normalisation} \\
\mbox{value} &\leftrightarrow \mbox{normal deduction}
\end{align*}
To reap more benefits from the connection we tapped
into, it might be a good idea to look at other connectives from intuitionistic
logic. Conjunctions have an obvious interpretation as pairs of terms,
disjunctions can be seen as structures containing at least one value,
corresponding to which disjuncts are true. Leaving the domain of propositional
logic we might look at universal quantifiers, relating to type-agnostic
functions that may help easing some of the complications we saw the introduction
of types bringing to lambda terms in \ref{section:ski}.

Considering then that there are bigger logics than intuitionistic logic for
which correspondences to computational systems can be established, it becomes
clear that opportunities for further investigations are plenty.
