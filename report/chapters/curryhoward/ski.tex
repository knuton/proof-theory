\section{Combinators and Axioms}

The main result of the previous section is known under various names, among them
the \textit{formulas-as-types} and \textit{proofs-as-programs} interpretations.
When we showed three exemplary deductions of axioms of propositional logic in
\ref{sec:naturaldeduction}, we took care to sneak in something that would make
sense in the light of these interpretations. The three axioms that were deduced
correspond to typed instances of three \textit{combinators} that are known as
$\mathsf{I}$, $\mathsf{K}$ and $\mathsf{S}$ in the literature:
\begin{align*}
\mathsf{I} &=
  (\lambda u^A.u)^A \\
\mathsf{K} &=
  (\lambda u^A.
     (\lambda v^B.u)
  )^{A \to B \to A} \\
\mathsf{S} &=
  (\lambda u^{A \to B \to C}.
     (\lambda w^{A \to B}.
       (\lambda v^A.(uv)(wv))
     )
  )^{(A \to B \to C) \to ((A \to B) \to (A \to C))}
\end{align*}
The system consisting of the $\mathsf{SKI}$ combinators is a well-studied Turing
complete system. While $\mathsf{SKI}$ due to its bare manner is of theoretical
interest only, it is not a minimal system. Instead, already the system
$\mathsf{SK}$ is Turing complete, as $\mathsf{I}$ can be obtained by combining
$\mathsf{S}$ and $\mathsf{K}$. Looking at the combinators as given above, we see
a problem with this. We can neither apply $\mathsf{S}$ to $\mathsf{K}$ nor
$\mathsf{K}$ to $\mathsf{S}$, because the types don't match.

Unless we treat each of the combinators as a schema to be instantiated by
assigning types to the type meta-variables occuring in the expression, we will
have to construct appropriate combinators for specific types. Because of the
workings of \implnpi\ we know that we could always do that by using the
appropriate simple types from $\Pi$ as assumptions in the appropriate places. We
can therefore justify the meta-rule of using the combinators above as schemas.

We will now combine appropriate instances of $\mathsf{S}$ and $\mathsf{K}$ to
yield $\mathsf{I}$. To better manage the size of the expressions, we will omit
types where unproblematic and use variables for terms.
\begin{align*}
\mathsf{S} &=
(\lambda u^{(A \to (B \to A) \to A) \to (A \to (B \to A)) \to (A \to A)}.
  (\lambda w^{A \to B \to A}.
    (\lambda v^A.(uv)(wv))
  )
) \\
\mathsf{K_1} &=
(\lambda u_1^A.
  (\lambda v_1^{B \to A}.u)
) \\
\mathsf{K_2} &=
(\lambda u_2^A.
  (\lambda v_2^B.u)
) \\
\mathsf{(SK)K} &=_\beta
(
  (\lambda u^{(A \to (B \to A) \to A) \to (A \to (B \to A)) \to (A \to A)}.
    (\lambda w^{A \to B \to A}.
      (\lambda v^A.(uv)(wv))
    )
  )
  \mathsf{K_1}
)
\mathsf{K_2} \\
&=_\beta
(\lambda w^{A \to B \to A}.
  (\lambda v^A.(\mathsf{K_1}v)(wv))
)
\mathsf{K_2} \\
&=_\beta
(\lambda v^A.(\mathsf{K_1}v)(\mathsf{K_2}v)) \\
&=_\beta
(\lambda v^A.
  ((\lambda u_1^A.
    (\lambda v_1^{B \to A}.u)
  )v)
  (\mathsf{K_2}v)
) \\
&=_\beta
(\lambda v^A.
  (\lambda v_1^{B \to A}.v)
  (\mathsf{K_2}v)
) \\
&=_\beta
(\lambda v^A.
  (\lambda v_1^{B \to A}.v)
  ((\lambda u_2^A.
     (\lambda v_2^B.u_2)
  )v)
) \\
&=_\beta
(\lambda v^A.
  (\lambda v_1^{B \to A}.v)
  (\lambda v_2^B.v)
) \\
&=_\beta
(\lambda v^A.v) \\
&=
\mathsf{I}
\end{align*}
We were thus able to derive $\mathsf{I}$, given we identify terms that differ only in
the names of bound variables.

It would be interesting to see what transformations are being made on the
corresponding deduction trees while we following steps of $\beta$-reduction.
Alas, the terms above become very complicated when translated into
deductions. We will have to look at some simpler examples.
