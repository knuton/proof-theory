\section{The Isomorphism}
\label{section:bijection}

\begin{definition}
We define the notation $M^{A}$ for a simply-typed lambda term $M$ to mean that
$M$ contains the outermost type annotation $A$, i.e. $M^A \equiv M':A$.
\end{definition}

\begin{theorem}[Curry-Howard Isomorphism]
The natural deduction system \implnpi\ consisting of the set $\npiproofs$ of
deduction trees with implication introduction and implication elimination as
operations is isomorphic to the term calculus \stlambda\ consisting of the set
$\stlambdaterms$ of simply-typed lambda terms with functional abstraction and
function application as operations.
\end{theorem}

\begin{proof}
We give a bijection $\chmap: \npiproofs \to \stlambdaterms$ from the set $\npiproofs$
of \implnpi\ deduction trees to the set $\stlambdaterms$ of simply-typed lambda terms.
\begin{alignat*}{2}
\chmap(A^u) &:= u: A \\[6pt]
\chmap\left(
  \AxiomC{$\mathcal{D}_1$}
  \noLine
  \UnaryInfC{$A \to B$}
  \AxiomC{$\mathcal{D}_2$}
  \noLine
  \UnaryInfC{$A$}
  \RightLabel{$\to$E}
  \BinaryInfC{$B$}
  \DisplayProof
\right) &:= \chmap(\mathcal{D}_1) \chmap(\mathcal{D}_2): B \\[6pt]
\chmap\left(
  \AxiomC{$\mathcal{D}$}
  \noLine
  \UnaryInfC{$B$}
  \RightLabel{$\to$I, $u$}
  \UnaryInfC{$A \to B$}
  \DisplayProof
\right) &:= (\lambda u: A.\chmap(\mathcal{D})): A \to B
\end{alignat*}

\begin{proposition}
If $\mathcal{D}$ is a deduction of $A$, then $\chmap(\mathcal{D})$ has type $A$.
\end{proposition}

\begin{proof}
By induction on the complexity of $\mathcal{D}$.
\end{proof}

\noindent We additionally define a function $\stlambdaterms \to \npiproofs$,
suggestively named $\chmapanta$:
\begin{alignat*}{2}
\chmapanta(u: A) & := A^u \\[6pt]
\chmapanta((M^{A \to B})(N^A): B) &:=
  \AxiomC{$\chmapanta(M^{A \to B})$}
  \AxiomC{$\chmapanta(N^A)$}
  \RightLabel{$\to$E}
  \BinaryInfC{B}
  \DisplayProof \\[6pt]
\chmapanta((\lambda u:A.M^B):A \to B) &:=
  \AxiomC{$\chmapanta(M^B)$}
  \RightLabel{$\to$I, $u$}
  \UnaryInfC{$A \to B$}
  \DisplayProof
\end{alignat*}

\begin{proposition}
For any lambda term $M^A$, $\chmapanta(M^A)$ is a proof of $A$.
\end{proposition}

\begin{proof}
By induction on the complexity of lambda terms.
\end{proof}

\noindent We now show that $\chmap$ is a bijection by showing that
\begin{itemize}
\item[(i)] $\chmapanta \circ \chmap = id_{\npiproofs}$ and
\item[(ii)] $\chmap \circ \chmapanta = id_{\stlambdaterms}$.
\end{itemize}
The proof is by induction on the construction of deduction trees and lambda
terms respectively.\\
\\
\noindent (i) $\chmapanta \circ \chmap = id_{\npiproofs}$
% Base case
\[
\chmapanta(\chmap(A^u)) = \chmapanta(u:A) = A^u = id_{\npiproofs}(A^u)
\]
% Impl elim
\begin{align*}
\chmapanta\left(\chmap\left(
  \AxiomC{$\mathcal{D}_1$}
  \noLine
  \UnaryInfC{$A \to B$}
  \AxiomC{$\mathcal{D}_2$}
  \noLine
  \UnaryInfC{$A$}
  \RightLabel{$\to$E}
  \BinaryInfC{$B$}
  \DisplayProof
\right)\right)
& =
  \chmapanta(\chmap(\mathcal{D}_1) \chmap(\mathcal{D}_2): B)
  \tag{By def. $\chmap$} \\[6pt]
& =
  \AxiomC{$\chmapanta(\chmap(\mathcal{D}_1))$}
  \noLine
  \UnaryInfC{$A \to B$}
  \AxiomC{$\chmapanta(\chmap(\mathcal{D}_2))$}
  \noLine
  \UnaryInfC{$A$}
  \RightLabel{$\to$E}
  \BinaryInfC{B}
  \DisplayProof \tag{By def. $\chmapanta$} \\[6pt]
& =
  \AxiomC{$\mathcal{D}_1$}
  \noLine
  \UnaryInfC{$A \to B$}
  \AxiomC{$\mathcal{D}_2$}
  \noLine
  \UnaryInfC{$A$}
  \RightLabel{$\to$E}
  \BinaryInfC{$B$}
  \DisplayProof \tag{By induction hypothesis} \\[6pt]
& =
  id_{\npiproofs}\left(
  \AxiomC{$\mathcal{D}_1$}
  \noLine
  \UnaryInfC{$A \to B$}
  \AxiomC{$\mathcal{D}_2$}
  \noLine
  \UnaryInfC{$A$}
  \RightLabel{$\to$E}
  \BinaryInfC{$B$}
  \DisplayProof
  \right)
\end{align*}
% Impl intro
\begin{align*}
\chmapanta\left(\chmap\left(
  \AxiomC{$\mathcal{D}$}
  \noLine
  \UnaryInfC{$B$}
  \RightLabel{$\to$I, $u$}
  \UnaryInfC{$A \to B$}
  \DisplayProof
\right)\right)
& =
  \chmapanta((\lambda u: A.\chmap(\mathcal{D})): A \to B)
  \tag{By def. $\chmap$} \\[6pt]
& =
  \AxiomC{$\chmapanta(\chmap(\mathcal{D}))$}
  \noLine
  \UnaryInfC{$B$}
  \RightLabel{$\to$I, $u$}
  \UnaryInfC{$A \to B$}
  \DisplayProof \tag{By def. $\chmapanta$} \\[6pt]
& =
  \AxiomC{$\mathcal{D}$}
  \noLine
  \UnaryInfC{$B$}
  \RightLabel{$\to$I, $u$}
  \UnaryInfC{$A \to B$}
  \DisplayProof \tag{By induction hypothesis} \\[6pt]
& =
  id_{\npiproofs}\left(
  \AxiomC{$\mathcal{D}$}
  \noLine
  \UnaryInfC{$B$}
  \RightLabel{$\to$I, $u$}
  \UnaryInfC{$A \to B$}
  \DisplayProof
  \right)
\end{align*}

\noindent (ii) $\chmap \circ \chmapanta = id_{\stlambdaterms}$
% Base case
\begin{align*}
\chmap(\chmapanta(u:A)) & = \chmap(A^u) = u:A = id_{\stlambdaterms}(u:A)
\end{align*}
% Function application
\begin{align*}
\chmap(\chmapanta(M^{A \to B}N^A:B)) & =
  \chmap\left(
  \AxiomC{$\chmapanta(M^{A \to B})$}
  \noLine
  \UnaryInfC{$A \to B$}
  \AxiomC{$\chmapanta(N^A)$}
  \noLine
  \UnaryInfC{$A$}
  \RightLabel{$\to$E}
  \BinaryInfC{B}
  \DisplayProof
  \right) \tag{By def. $\chmapanta$}\\[4pt]
& =
  \chmap(\chmapanta(M^{A \to B}))\chmap(\chmapanta(N^A)): B
  \tag{By def.  $\chmap$}\\[6pt]
& =
  M^{A \to B}N^A:B
  \tag{By induction hypothesis}\\[6pt]
& =
  id_{\stlambdaterms}(M^{A \to B}N^A:B)
\end{align*}
% lambda abstraction
\begin{align*}
\chmap(\chmapanta((\lambda u:A.M^B):A \to B)) &=
  \chmap\left(
  \AxiomC{$\chmapanta(M^B)$}
  \noLine
  \UnaryInfC{$B$}
  \RightLabel{$\to$I, $u$}
  \UnaryInfC{$A \to B$}
  \DisplayProof
  \right) \tag{By def. $\chmapanta$} \\[6pt]
& =
  (\lambda u: A.\chmap(\chmapanta(M^B))): A \to B
  \tag{By def. $\chmap$} \\[6pt]
& =
  (\lambda u: A.M^B): A \to B
  \tag{By induction hypothesis} \\[6pt]
& =
  id_{\stlambdaterms}((\lambda u: A.M^B): A \to B)
\end{align*}

The correspondence of implication introduction with functional abstraction, and
implication elimination with function application is immediate from the
definition of $\chmap$ and $\chmapanta$ and omitted for brevity.
\end{proof}
