\section{Consequences of the Correspondence}

Having achieved the correspondence we can now start applying results
established in one camp to the other. There are two examples that we would like
to explore.

The normalisation results for \stlambda\ immediately imply equivalent notions
of weak and strong normalisation of deduction trees. In fact, this is not
surprising as the proof in \ref{thm:weaknorm} can be reused for proving the
weak normalisation of deductions in \implnpi\ if one replaces the notion of a
\emph{redex} with the notion of a \emph{cut-segment} in a deduction tree (and the
related concepts), but keeps the overall structure of the deduction the same.

What is perhaps more surprising is that one can proceed by using corollaries of
the Church-Rosser Theorem, such as uniqueness of normal forms
(\ref{cor:uniqnorm}) which imply that there exists a unique normal form of a
deduction tree as well and that any order of normalisation steps in a
deduction tree will produce a unique deduction tree in a finite number of steps.

Results on unprovability can be difficult to obtain, so an interesting
application is the following:

\begin{proposition}\implnpi\ is consistent.\end{proposition}
\begin{proof}
    Assume there is a closed deduction tree for $\bot$ in \implnpi. Then there
    exists an $M\!:\!\bot \in \Lambda_\Pi$ and by weak normalisation and
    uniqueness of types, there must exist an $N:\bot \betared^* N:\bot$ and $N$
    is in normal form. This means that either $N$ is of the form $\lambda x.
    N'$ (where $N'$ is in normal form), in which case $N$ would have type
    $N:\sigma\to\tau$, but this would mean that $\bot = \sigma\to\tau$, which
    is clearly false.  Otherwise, $N$ could be of the form $x N_1 \ldots N_m$
    (where $N_i$'s are in normal-form), but this is again not possible since
    then $x$ is a free variable, but $FV(M:\bot) = FV(N:\bot) = \{\}$.
\end{proof}
