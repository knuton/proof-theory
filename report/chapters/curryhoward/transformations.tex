\section{Transformations of Prooftrees}

Since by \ref{cor:uniquenessoftypes} we know that the types of any terms, that
are equivalent under $\to_\beta$, coincide, we can say that the deductions
corresponding to $(\mathsf{SK_1})\mathsf{K_2}$ and $\mathsf{I}$ must possess the
same conclusion. Clearly, the deduction corresponding to $\mathsf{I}$ will be
more economical. We can also see that the term $\mathsf{I}$ is in normal form.
This example could tempt us to think that the normal form is a way of
economically notating a term or that proofs corresponding to normal forms
are more concise. But this is not generally the case.\\
\\
Consider the term
\[
(\lambda v^{A \to B}.
  v(u^{(A \to B) \to A}v)
)(
  w^{(C \to D) \to (A \to B)}(\lambda y^C.x^D)
)
\]
and its one-step conversum
\[
(w^{(C \to D) \to (A \to B)}(\lambda y^C.x^D))(u^{(A \to B) \to A}w^{(C \to D) \to (A \to B)}(\lambda y^C.x^D))
\]
with corresponding deduction trees
\begin{prooftree}
\scriptsize
\implelim{%
  \implintro{%
    \implelim{\AxiomC{$A \to B^v$}}{%
      \implelim{\AxiomC{$(A \to B) \to A^u$}}{\AxiomC{$A \to B^v$}}{A}
    }{$B$}
  }{v}{$(A \to B) \to B$}
}{%
  \implelim{\AxiomC{$(C \to D) \to (A \to B)^w$}}{%
    \implintro{\AxiomC{$D^x$}}{y}{$C \to D$}
  }{$A \to B$}
}{$B$}
\end{prooftree}
and
\begin{prooftree}
\scriptsize
\implelim{%
  \implelim{\AxiomC{$(C \to D) \to (A \to B)^w$}}{%
    \implintro{\AxiomC{$D^x$}}{y}{$C \to D$}
  }{$A \to B$}
}{%
  \implelim{%
    \AxiomC{$(A \to B) \to A^u$}
  }{%
    \implelim{\AxiomC{$(C \to D) \to (A \to B)^w$}}{%
      \implintro{\AxiomC{$D^x$}}{y}{$C \to D$}
    }{$A \to B$}
  }{A}
}{$B$}
\end{prooftree}
. Both do not seem like improvements in use of space, ink or memory. And looking
at it from the perspective of lambda functions this makes intuitive sense:
Functional abstraction allows us to assign a name to a term and then refer to it
multiple times without writing out the full corresponding expression each time.
Analogously, the use of detours in deductions allows us to appeal to
another proof multiple times without spelling it out over and over. Then, if
necessary, we can instantiate the proof at the appropriate positions. In hope to
transfer more ideas from one structure to the other, we will look at the
relation of normalisation in deductions and lambda terms in some detail.
