\section{The Natural Deduction System \implnpi}
\label{sec:naturaldeduction}

% I intend to add or redo passages as this becomes necessary for later sections

In this section we will briefly review natural deduction systems. In particular
we will look at the natural deduction system \implnpi\ for intuitionistic
propositional logic with implication as the only connective. We call the
corresponding language $\limpl$, its grammar is given by
$$\limpl ::=\ \perp |\ P\ |\ \limpl \to \limpl,$$
where $P$ is any of countably many propositional variable names.

We will conceptualize proofs as trees of formulas and proving as constructing
such trees. A formula $A$ is said to be \textit{deducible} in \implnpi\ if the
rules of the system allow the construction of a \textit{deduction tree} with its
root node being the conclusion $A$. Any non-leaf node $q$ possesses as its
children $n$ nodes that are the premisses of a rule application of which $q$ is
the conclusion. The leaf nodes of a deduction tree are its \textit{assumptions},
each labelled by a variable name as its marker. Multiple occurrences of the same
formula as assumptions with the same marker are said to compose an
\textit{assumption class}. If the marker labelling an assumption class is
mentioned in a rule application in the deduction tree, the assumption class is
called \textit{closed}, otherwise \textit{open}. The open assumptions in a
deduction tree are the assumptions of the deduction.

The rules for \implnpi\ allow as the simplest deduction of any $\limpl$ formula
$A$ the proof of $A$ from open assumption $A$ as the one-node deduction tree
consisting only of $A$. More complex deductions can be constructed by combining
any number of deduction trees by means of the system's only two rules, the
\textit{implication introduction rule} $\to$I and the
\textit{implication elimination rule} $\to$E:

\begin{center}
\AxiomC{$[A]^u$}
\noLine
\UnaryInfC{$\mathcal{D}_1$}
\noLine
\UnaryInfC{$B$}
\RightLabel{$\to$I, $u$}
\UnaryInfC{$A \to B$}
\DisplayProof
% TODO A bit more horizontal space would be nice
\AxiomC{$\mathcal{D}_1$}
\noLine
\UnaryInfC{$A \to B$}
\AxiomC{$\mathcal{D}_2$}
\noLine
\UnaryInfC{$A$}
\RightLabel{$\to$E}
\BinaryInfC{$B$}
\DisplayProof
\end{center}

In this notation, $\mathcal{D}_1$ and $\mathcal{D}_2$ are deduction trees. The
lines above and below a deduction tree are understood to be part of the tree
unless separated by a horizontal line. A horizontal line signifies a rule
application, with the name of the rule and optionally a variable name appearing
right of the line. $A$ and $B$ are formulas of $\mathcal{L}_\to$. $[A]^u$
denotes the assumption class of all occurrences of $A$ labelled with variable
name $u$. We see that $\to$I closes an assumption class, which we allow to
consist of zero occurrences of $A$.

% TODO Proper reference
This should clarify what we take ``\implnpi'' to mean, for details we refer to
Troelstra-Schwichtenberg whose general formalism we follow.
