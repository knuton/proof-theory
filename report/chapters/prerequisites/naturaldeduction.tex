\section{The Natural Deduction System \implnpi}
\label{sec:naturaldeduction}

% I intend to add or redo passages as this becomes necessary for later sections

In this section we will briefly review natural deduction systems. In particular
we will look at the natural deduction system \implnpi\ for intuitionistic
propositional logic with implication as the only connective. We call the
corresponding language $\limpl$, its grammar is given by
$$\limpl ::=\ \perp |\ P\ |\ \limpl \to \limpl,$$
where $P$ is any of countably many propositional variable names.

We will conceptualize proofs as trees of formulas and proving as constructing
such trees. A formula $A$ is said to be \textit{deducible} in \implnpi\ if the
rules of the system allow the construction of a \textit{deduction tree} with its
root node being the conclusion $A$. We would like for every node of the
deduction tree to be proven by the subtree of which it is the conclusion. This
is easy for any non-leaf node $q$ if it possesses as its children $n$ nodes that
are the premises of a rule application of which $q$ is the conclusion. But we
require deduction trees to be finite structures and thus will have to deal with
leaf nodes that do not have any children. We will in this case simply assume to
possess some proof $u$ of the node at hand. To keep track of the purported
proof, we will label the node with the name $u$ as a marker. We disallow
occurrences of different formulas to be labeled with the same marker, noting
that it does not make sense to purport $u$ to deduce two distinct formulas $A$
and $B$. We call the leaf nodes of a deduction tree its \textit{assumptions}.
Multiple occurrences of the same formula as assumptions with the same marker are
said to compose an \textit{assumption class}. If the marker labeling an
assumption class is mentioned in a rule application in the deduction tree, the
assumption class is called \textit{closed}, otherwise \textit{open}. The open
assumptions in a deduction tree are the assumptions of the deduction.

The rules for \implnpi\ allow as the simplest deduction of any $\limpl$ formula
$A$ the proof of $A$ from open assumption $A$ as the one-node deduction tree
consisting only of $A$. This amounts to the plausible claim that if we possessed
a proof of $A$, we could indeed proof $A$. More complex deductions can be
constructed by combining any number of deduction trees by means of the system's
only two rules, the \textit{implication introduction rule} $\to$I and the
\textit{implication elimination rule} $\to$E:

\begin{center}
\AxiomC{$[A]^u$}
\noLine
\UnaryInfC{$\mathcal{D}_1$}
\noLine
\UnaryInfC{$B$}
\RightLabel{$\to$I, $u$}
\UnaryInfC{$A \to B$}
\DisplayProof
\hspace{2.4em}
\AxiomC{$\mathcal{D}_1$}
\noLine
\UnaryInfC{$A \to B$}
\AxiomC{$\mathcal{D}_2$}
\noLine
\UnaryInfC{$A$}
\RightLabel{$\to$E}
\BinaryInfC{$B$}
\DisplayProof
\end{center}

In this notation, $\mathcal{D}_1$ and $\mathcal{D}_2$ are deduction trees. The
lines above and below a deduction tree are understood to be part of the tree
unless separated by a horizontal line. A horizontal line signifies a rule
application, with the name of the rule and optionally a marker name appearing
right of the line. Note that the rule for implication elimination corresponds to
a \textit{partial} function on the set of deduction trees, as only deduction
trees with matching conclusions can be combined. $A$ and $B$ are formulas of
$\mathcal{L}_\to$. $[A]^u$ denotes the assumption class of all occurrences of
$A$ labeled with marker $u$.  We see that $\to$I closes an assumption class,
which we allow to consist of zero occurrences of $A$.

This should clarify what we take ``\implnpi'' to mean, for details we refer to
\parencite{basicprooftheory} whose general formalism we follow.

We will now produce exemplary proofs of three axioms of propositional logic:

\begin{prooftree}
\AxiomC{$A^u$}
\RightLabel{$\to$I, $u$}
\UnaryInfC{$A \to A$}
\end{prooftree}

\begin{prooftree}
\AxiomC{$A^u$}
\RightLabel{$\to$I, $v$}
\UnaryInfC{$B \to A$}
\RightLabel{$\to$I, $u$}
\UnaryInfC{$A \to (B \to A)$}
\end{prooftree}

\begin{prooftree}
\AxiomC{$A \to (B \to C)^u$}
\AxiomC{$A^v$}
\RightLabel{$\to$E}
\BinaryInfC{$B \to C$}
\AxiomC{$A \to B^w$}
\AxiomC{$A^v$}
\RightLabel{$\to$E}
\BinaryInfC{$B$}
\RightLabel{$\to$E}
\BinaryInfC{$C$}
\RightLabel{$\to$I, $v$}
\UnaryInfC{$A \to C$}
\RightLabel{$\to$I, $w$}
\UnaryInfC{$(A \to B) \to (A \to C)$}
\RightLabel{$\to$I, $u$}
\UnaryInfC{$(A \to (B \to C)) \to ((A \to B) \to (A \to C))$}
\end{prooftree}
