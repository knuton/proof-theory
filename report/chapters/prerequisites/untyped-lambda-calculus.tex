\section{The Untyped $\lambda$-calculus}

The language of untyped $\lambda$-calculus is surprisingly simple.\footnote{Here we mostly follow the style used in \smartcite{introtolambda}.}

We begin with an infinite set of variables $V = \{v, v', v'', \ldots\}$ from
which we construct the set of $\lambda$-terms which we shall call $\Lambda$.

\begin{definition}[Untyped $\lambda$-calculus]\label{def:untyped-lambda-calc}
We define the set $\Lambda$ recursively by:
\begin{align*}
    x \in V & \Rightarrow x \in \Lambda &
        \text{variables} \\
    M \in \Lambda,\, x \in V & \Rightarrow (\lambda x M) \in \Lambda &
        \text{$\lambda$-abstraction} \\
    M, N \in \Lambda & \Rightarrow (M N) \in \Lambda &
        \text{function application}
\end{align*}
\end{definition}

\emph{Note}: We shall always use lower case letters to denote \emph{variables} and
uppercase letters to denote arbitrary $\lambda$-terms.

Intuitively, we think of $\lambda$-abstractions as functions, where the meaning
of the word `function' lies somewhere in between our understanding of functions as
part of mathematics and functions as part of computer programs. We shall expand
on this idea in later parts.

In the same spirit, we think of $(M N)$ as ``function $M$ applied to $N$''.

To avoid clutter with excessive bracketing, we assume the usual conventions:
\begin{align*}
    \lambda x_1 x_2 \ldots x_n . M &\equiv
        (\lambda x_1 (\lambda x_2 (\cdots (\lambda x_n M)) \\
    F M N &\equiv ((FM)N)
\end{align*}

The $\lambda$ operator, in the same way as quantifiers in First-Order Logic,
binds the variable in front of it, hence we have the definition:
\begin{definition}[Free variables]
    The set of free variables of $M \in \Lambda$, denoted $FV(M)$ is defined
    recursively as:
        \begin{align*}
            FV(x) &= \{x\} \\
            FV(\lambda x M) &= FV(M) \setminus \{x\} \\
            FV(M N) &= FV(M) \cup FV(N)
        \end{align*}
\end{definition}

Conversely, we say that a variable $x$ occurring in $M$ is \emph{bound} if it is
not in $FV(M)$. We also say that $M$ is \emph{closed} iff $FV(M) = \emptyset$.

%% God damn it, there is something wrong with this substitution stuff, but I
%% can't put my finger on it.
\begin{definition}[Substitution]
    Let $M,\, N \in \Lambda$, $x, y\in V$. We denote the
    substitution of (free occurances of) $x$ in $M$ by $N$ as $M[N/x]$ and
    define it as follows (here $x \neq y$):
    \begin{align*}
        x[N/x] &= N & \\
        y[N/x] &= y & \\
        (P Q)[N/x] &= P[N/x]\; Q[N/x] & \\
        (\lambda x. P)[N/x] &= \lambda x.P & \\
        (\lambda y. P)[N/x] &= \lambda y.P[N/x] & y \notin FV(N)
    \end{align*}
\end{definition}

%%% @todo: make this more formal, review if I have skipped too much.
%\begin{definition}[$\alpha$-renaming]
%Since our intuition is to think of $\lambda$-calculus as a way to express
%computation, the variable names should not go in the way, therefore we identify
%two $\lambda$-terms if they only differ in the names of bound variables, e.g.
%$\lambda x x \equiv \lambda y y$. Furthermore, we can replace any subterm of a
%$\lambda$-term by one which has been defined as equivalent.
%\end{definition}

Since $\lambda$-calculus is supposed to express computable terms, we need to
have a way of evaluating them. This is achieved via something called
$\beta$-reduction. One can view $\beta$-reduction as yet another scheme for
replacements (such as $\alpha$-conversion).

\begin{definition}[$\beta$-reduction] We define $\beta$-reduction as the
opperation $\betared$, such that:
\[ (\lambda x M) N \betared M[N/x] \]

It is important to note that $\betared$ induces a relation on $\Lambda$. We
want to be able to reduce subterms of any $\lambda$-term so we require that
$\betared$ is closed under the following rules:
\begin{align*}
    P \betared P' &\Rightarrow
        \begin{cases}
            \forall x \in V: & \lambda x.P \betared \lambda x. P' \\
            \forall Z \in \Lambda: & (P\, Z) \betared (P'\, Z) \\
            \forall Z \in \Lambda: & (Z\, P) \betared (Z\, P')
        \end{cases}
\end{align*}

\end{definition}

%% @todo: maybe this part is BS/not that much related to our paper, so I
%% don't know why I actually wrote this down
At this point we can see how functions (i.e. $\lambda$-abstractions) are
similar and differ from $\lambda$-calculus functions in mathematics:
Consider the identity function $f(x) = x$ and the equivalent $\lambda x. x$

An application of $f$ to $3$ in mathematics would correspond to: $f(3) = 3$.

Although $3$ is not an element of $\Lambda$, we can informally write:
    \[ (\lambda x. x\, 3) \betared 3 \]

So in both cases what we did is just replacing the variable $x$ with $3$. Now
for the unsimilar part, since we can apply the $\lambda$-abstraction to
\emph{any} term, we can apply it to, say, another $\lambda$-abstraction:
    \[ (\lambda x. x\, \lambda y. 1) \betared \lambda y. 1 \]

$\lambda y. 1$ can be thought of as a constant function which always returns
$1$.  In mathematics, this would correspond to something like: $g(x) = 1$.  The
difference arises when we try to apply $f$ on $g$. In mathematics, one cannot
write $f(g) = g$, we can only do explicit function composition $(f \circ g)(x) =
g(x) = 1$. This is because functions are not first class citizens in the
language. But fortunately they are in $\lambda$-calculus, which allows us to do
fun things.

Consider the abstraction $\lambda x.\; x\, x$, call it $\bm{D}$ for double.
What happens if we apply $\bm{D}$ to itself?
    \[ (\bm{D} \bm{D}) = (\lambda x.\; x\, x\: \bm{D})
        \betared (\bm{D} \bm{D}) \]

So it seems that evaluating $(\bm{D} \bm{D})$ results in itself, which means
that if we were to write a program that would try to evaluate it, it would get
stuck in an infinite loop. We can even think of worse examples by taking
$\bm{T} = \lambda x.\; x\, x\,x$ and trying to evaluate $\bm{T}\, \bm{T}$.
Again, this hints on the similarity with computation (via non-terminating
programs).

\begin{definition}[$\beta$-redex]
    A term of the form $(\lambda x. M) N$ is called a ($\beta$-)\emph{redex}.
\end{definition}
\begin{definition}[$\beta$-normal form]
    A term which can not be (further) $\beta$-reduced is said it to be in
    ($\beta$)-\emph{normal form}.
\end{definition}

Of a particular interest are iterated applications of $\betared$ and the
reflexive closure of the resulting relation:
\begin{definition}[$\betared^*$]
    $\betared^*$ is the transitive closure of $\betared$.
\end{definition}

\begin{definition}[$\leftrightarrow_\beta^*$]
    $\leftrightarrow_\beta^*$ is the reflexive closure of $\betared^*$.
\end{definition}

\begin{definition}[$=_\beta$]
    $=_\beta$ is the symmetric closure of $\leftrightarrow_\beta^*$, or in
    other words the equivalence relation induced by $\betared$.
\end{definition}

If for two $\lambda$-terms we have that $M =_\beta N$ then our intuition is
that they compute the same thing.

A crucial property of this relation is exposed by the following theorem.
\begin{theorem}[Church-Rosser Theorem]
    Let $M \in \Lambda$. If $M \betared^* M_1$ and $M \betared^* M_2$ then
    there exists $N$ s.t. $M_1 \betared^* N$ and $M_2 \betared^* N$.
\end{theorem}
\begin{proof}Omitted.\end{proof}

Here's a pretty diagram to illustrate the theorem:
\begin{diagram}
             &                 & M &                 &       \\
             & \ldTo_\beta     &   & \rdTo_\beta     &       \\
         M_1 &                 &   &                 &  M_2  \\
             & \rdDotsto_\beta &   & \ldDotsto_\beta &       \\
             &                 & N &                 &
\end{diagram}

\begin{corollary}[Uniqueness of normal form]
    If $M \betared^* N_1$ and $M \betared^* N_2$ and $N_1$ and $N_2$ are in
    normal form, then $N_1 = N_2$.
\end{corollary}

\begin{corollary}
    If $M_1 \betared^* N_1$ and $M_2 \betared^* N_2$, $N_1 \neq N_2$ and $N_1, N_2$
    are in normal form, then $M_1 \neq_\beta M_2$.
\end{corollary}

