\section{Simply typed $\lambda$-calculus}

We shall now extend the language of untyped $\lambda$-calculus by
adding \emph{types} for the terms. The notion of types is very natural to
someone with a background in programming, but perhaps not so much to someone
from the background of mathematics.

In mathematics the role of types is played by the domain and co-domain of a
function, e.g. when we write down $f: \mathbb{Z} \rightarrow \mathbb{N}$ we
mean that the function $f$ transforms elements of type $\mathbb{Z}$ into
elements of type $\mathbb{N}$ and thus has the type $\mathbb{Z} \rightarrow
\mathbb{N}$. The number $-3$ is thus of type $\mathbb{Z}$ and and a transformation like
$x \mapsto x^2$ transforms it to $9$ which is of type $\mathbb{N}$.

Again, the language of simply typed $\lambda$-calculus is a minimalistic
extension which turns out to be very powerful. In defining it we shall follow
the style of Church i.e. use explicit type annotations for the variables in lambda
abstractions: $\lambda x. M$ will become $\lambda x:\sigma. M$ where
$\sigma$ is the type.

\begin{definition}
The language of simply typed $\lambda$-calculus is built from:
    \begin{enumerate}[(i)]
        \item A countably infinite set of type variables $U$.
        \item The set of simple types $\Pi$ which is defined recursively as:
        \begin{align*}
            \alpha \in U &\Rightarrow \alpha \in \Pi \\
            \sigma, \tau \in \Pi &\Rightarrow (\sigma \to \tau) \in \Pi
        \end{align*}
            \emph{Convention}. $\to$ associates to the right. This convention
            allows us to drop parenthesis.
        \item We have to modify our original set of $\lambda$-terms $\Lambda$
            by adding explicit type annotations, so we define the set
            $\Lambda_\Pi$ of (explicitly annotated) $\lambda$-terms as:
                \begin{align*}
                    x \in V & \Rightarrow x \in \Lambda_\Pi &
                        \text{variables} \\
                    M \in \Lambda_\Pi,\, x \in V, \sigma \in \Pi & \Rightarrow
                    (\lambda x\!:\!\sigma\; M) \in \Lambda_\Pi &
                        \text{$\lambda$-abstraction} \\
                    M, N \in \Lambda_\Pi & \Rightarrow (M N) \in \Lambda_\Pi &
                        \text{function application}
                \end{align*}
                Here $V$ is the same set of variables as in definition
                \ref{def:untyped-lambda-calc}.
        \item A \emph{context} is a set $\Gamma$ of pairs $x_i: \tau_i$, where $x_i \in
        \Lambda$, $\tau_i \in \Pi$ and the $x_i$'s are unique. We denote $C$ as
        the set of \emph{all} contexts.
        \item The \emph{domain} of a context $\Gamma$ is defined as:
            $dom(\Gamma) = \{x\,|\, x:\tau \in \Gamma \}$
        \item The \emph{range} of a context $\Gamma$ is defined as:
            $|\Gamma| = \{ \tau \,|\, x:\tau \in \Gamma\}$
        \item Most importantly, we introduce \emph{typability} relations (or
        \emph{rules}):

        \texttt{(TODO: Add explanation of commas in the sequents.)}

        \texttt{(TODO: How to horizontally align bussproofs?)}

        Identity: \begin{prooftree}
                        \AxiomC{}
                        \UnaryInfC{$\Gamma, x:\tau \vdash x:\tau$}
                        \end{prooftree}
        Abstraction: \begin{prooftree}
                    \AxiomC{$\Gamma, x:\sigma \vdash M:\tau$}
                    \UnaryInfC{$\Gamma \vdash (\lambda x\!:\!\sigma\, M):\sigma \to \tau$}
                \end{prooftree}

        Application: \begin{prooftree}
            \AxiomC{$\Gamma \vdash M:\sigma \to \tau$}
            \AxiomC{$\Gamma \vdash N:\sigma$}
            \BinaryInfC{$\Gamma \vdash (M\; N): \tau$}
        \end{prooftree}
        \end{enumerate}
\end{definition}


\begin{lemma}[Generation lemma]{\ }
\begin{enumerate}[(i)]
        \item $\Gamma \vdash x:\tau$ implies $x:\tau \in \Gamma$
        \item $\Gamma \vdash (M N):\tau$ implies that there exists a type
            $\tau$ s.t. $\Gamma \vdash M: \sigma \to \tau$ and $\Gamma \vdash
            N:\sigma$.
        \item $\Gamma \vdash \lambda x\!:\!\sigma\; M:\rho$ implies that $\rho = \sigma
            \to \tau$ and $\Gamma, x:\sigma \vdash M:\tau$.
    \end{enumerate}
\end{lemma}
\begin{proof}
    All of the proofs follow by induction on the length of the derivation, so we
    shall only prove it for case \emph{(iii)}. \texttt{TRIVIAL, HAHA. (@todo)}
\end{proof}

\begin{proposition}[Subject reduction]
    $\Gamma \vdash M:\sigma$ and $M \betared N$ implies $\Gamma \vdash
    N:\sigma$.
\end{proposition}
\begin{proof}
    \texttt{TBD.}
\end{proof}

\begin{proposition}[Uniqueness of types] {\ }
    \begin{enumerate}[(i)]
        \item $\Gamma \vdash M:\sigma$ and $\Gamma \vdash M:\tau$ implies
            $\sigma = \tau$.
        \item $\Gamma \vdash M:\sigma$ and $\Gamma \vdash N:\tau$ and $M =_\beta
            N$, then $\sigma = \tau$.
    \end{enumerate}
\end{proposition}
\begin{proof}
    \texttt{TBD.}
\end{proof}

