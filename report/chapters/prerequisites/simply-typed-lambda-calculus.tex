\section{Simply typed $\lambda$-calculus}

We now consider a \emph{typed} version of $\lambda$-calculus. The notion of
types is very natural to someone with a background in programming, but perhaps
not so much to someone from the background of mathematics.

In mathematics the role of types is played by the domain and co-domain of a
function, e.g. when we write down $f: \mathbb{Z} \rightarrow \mathbb{N}$ we
mean that the function $f$ transforms elements of type $\mathbb{Z}$ into
elements of type $\mathbb{N}$ and thus has the type $\mathbb{Z} \rightarrow
\mathbb{N}$. The number $-3$ is thus of type $\mathbb{Z}$ and a transformation like
$x \mapsto x^2$ transforms it to $9$ which is of type $\mathbb{N}$.

We shall follow the so-called Church style of typed $\lambda$-calculus, which
uses explicit type annotations for the variable immediatelly following
$\lambda$ in $\lambda$-abstractions, i.e. we shall write $\lambda
x\!:\!\sigma\; M$ and say that here $x$ is of type $\sigma$. As we will see,
this approach has the nice property of uniquely fixing the type of
any\footnote{
    In order to be precise, we should say ``of any \emph{typable}''
    term, but we only define this notion later on.}
$\lambda$ expression.

We will also on the syntactic level restrict our language to \emph{typable}
terms in order to be able to speak uniformly about arbitrary terms in the
language.

\begin{definition}
\label{def:stlambda}
The language of simply typed $\lambda$-calculus is built from:
    \begin{enumerate}[(i)]
        \item A countably infinite set of type variables $U$.
        \item The set of simple types $\Pi$ which is defined recursively as:
        \begin{align*}
            \alpha \in U &\Rightarrow \alpha \in \Pi \\
            \sigma, \tau \in \Pi &\Rightarrow (\sigma \to \tau) \in \Pi
        \end{align*}
            \emph{Convention.} $\to$ associates to the right. This convention
            allows us to drop parentheses.
        \item The set $V_\Pi$ of (typed) variables:
            \[ x \in V,\, \sigma \in \Pi \Rightarrow x\!:\!\sigma \in V_\Pi \]
                Here $V$ is the same set of variables as in definition
                \ref{def:untyped-lambda-calc}.
        \item The set $\Lambda_\Pi$ of (typed) $\lambda$-terms as:
                \begin{align*}
                    x\!:\!\tau \in V_\Pi
                        &\Rightarrow x\!:\!\tau \in \Lambda_\Pi &
                            \text{(typed) variables} \\
                    M\!:\!\tau \in \Lambda_\Pi,\, x\!:\!\sigma \in V_\Pi
                        &\Rightarrow (\lambda x\!:\!\sigma\; M)\!:\!(\sigma \to
                        \tau) \in \Lambda_\Pi &
                            \text{$\lambda$-abstraction} \\
                    M\!:\!(\sigma \to \tau),\, N\!:\!\sigma \in \Lambda_\Pi
                        &\Rightarrow (M\, N)\!:\!\tau \in \Lambda_\Pi &
                            \text{function application}
                \end{align*}
            \emph{Convention.} We will occasionally drop the type annotations
            when they are reconstructable from the rest of the annotations, for
            example bound variables in $\lambda$-abstractions are obviously of
            the same type.

        \end{enumerate}
\end{definition}
\emph{Note 1.}
    Observe that in the construction of $\Lambda_\Pi$ we are actually coupling the
    term construction with the typability rules for the types. This removes the
    need for explicit typability rules because we ensure that all of our terms
    are well-typed be definition.

\emph{Note 2.}
    The set $\Lambda_\Pi$ differs in a crucial way from the set of
    untyped terms $\Lambda$ in def.  \ref{def:untyped-lambda-calc}.
    Namely, we only allow application of $\lambda$-abstractions on a
    term with a matching type.  Given the intuition that
    $\lambda$-abstractions are functions, this restriction feels
    reasonable (since the function should be applied only on the
    `correct' type of input), but in fact this removes fixed-point
    combinators and the possibility to express recursion within this
    language. The reason for it is that our type language is not
    expressive enough for the notion of recursion. For example, it's
    not possible to assign any type to the expression $\lambda x.\; x\,
    x$. Nevertheless, this is not a particular problem for us, since in
    the context of CHI we only care about the well-typed terms.
    %% @todo: get references for these wonderful claims

The definitions of \emph{free variables}, \emph{substitution} and
$\beta$-reduction remain the same as in the case for untyped
$\lambda$-calculus, we only need to replace all untyped term occurrences with
explicitly typed ones, hence Church-Rosser theorem and it's colloraries hold
too. Furthermore, it's worth noting that it only makes sense to talk about
$\beta$-reduction of \emph{a} term in $\Lambda_\Pi$, i.e. $(M\, N) \in
\Lambda_\Pi$, but not of $(M\!:\!\sigma\; N\!:\!\tau)$, where $M:\sigma, N:\tau
\in \Lambda_\Pi$ since $(M\!:\!\sigma\; N\!:\!\tau)$ might not be well-typed
and hence not part of our language, so it doesn't make sense to talk about
$\beta$-redutions of it.

\begin{proposition}[Subject reduction/expansion]
    $M:\sigma \betared N:\tau$ implies $\sigma = \tau$.
\end{proposition}
\begin{proof}
    By induction on the complexity of the terms.
\end{proof}

\begin{corollary}[Uniqueness of types] {\ }
    If $M:\sigma =_\beta N:\tau$ then $\sigma = \tau$.
\end{corollary}

Some very important results on \emph{normalisation} can be established in
Church style simply typed $\lambda$-calculus.
\begin{theorem}[Weak normalisation]
    For any $M:\tau \in \Lambda_\Pi$, there exists a reduction strategy which
    'terminates' in a finite number of steps, i.e. there exists a sequence
    $M:\tau \betared M_1 \betared \ldots \betared M_n:\tau$, where $M_n:\tau$
    is in normal form.
\end{theorem}

Note that this does \emph{not} hold for untyped $\lambda$ calculus, since we
have seen that for example the term $(\lambda x.\, x x\;\; \lambda x. x x)$
cannot be reduced to a normal form. For any type assignment, this term is not
an element of $\Lambda_\Pi$ (this follows easily from the subject reduction
property), so this does not contradict the theorem.

\begin{proof}
    Here we state the standard proof due to Turing and Prawitz.
    Firstly, observe that a type can be viewed as a binary tree (type variables
    are leaves and $\to$ is an edge) and let $d:\Pi \to \mathbb{N}$ be the
    depth of the corresponding tree (we define the depth of a leave to be $0$).
    Furthermore, define the depth of a redex $(M\!:\!\sigma\; N\!:\!\tau)$ as
    $d(\sigma)$. If a term is in normal form we define it's depth to be $0$.
    Now for each $M \in \Lambda_\Pi$ we define:
    \begin{align*}
        m(M) &= (0, 0) \text{, if $M$ is in normal form } \\
        m(M) &= (d_{max}(M), r(M)) \\
    \shortintertext{where}
        d_{max}(M) &= \max\{d(R)\, |\, \text{R is a redex in $M$}\} \\
    \shortintertext{and}
        r(M) &= \text{ number of maximal redexes in $M$ (i.e. of depth $d_{max}(M))$}
    \end{align*}

    We prove the result by induction on lexicographically ordered tuples $m(M)$.
    For the base case of $m(M) = (0, 0)$ we know that $M$ is in normal form and
    there is nothing to prove.
    Assume the result holds for all $M$ such that $m(M) < (n, m)$. Let $m(M)$
    be such that $m(M) = (n, m)$. Obviously $M$ is not in normal form. We will
    now describe a strategy of $\beta$-reducing $M$ to $N$ so that either the
    number of maximal redexes decreases or $d_{max}(N) < d_{max}(M)$. Pick the
    rightmost maximal redex $R$ in $M$. Let $N$ be
    obtained after applying $\beta$-reduction on $M$ to $R$. We claim that this
    reduction can only introduce new redexes of strictly smaller depth than
    $d_{max}(M)$. We distinguish two cases:
    \begin{enumerate}[a)]
        \item One possibility of introducing more redexes is when in the
        redex $R = (\lambda x.\, P)\; Q$ there is more than one free occurance
        of $x$ in $P$ and we are potentially copying redexes from $Q$, but we
        are safe because we know that they can only be of lower depth, since
        redexes in $Q$ are to the right of $R$ and hence cannot be maximal
        \item The other possibility is we construct new redexes, which can
        happen in three ways:
            \begin{enumerate}[1)]
                \item $R$ is of the form $(\lambda x\!:\!(\sigma\to\tau).
                    \ldots x P \ldots) (\lambda y. Q) \betared
                        (\ldots (\lambda y. Q) P \ldots )$, then
                        $d_{max}((\lambda y. Q) P) < d_{max}(N)$

                \item $R$ appears in $M$ as $(R\; Q)$ and
                after reduction becomes a new redex of the form $(\lambda x
                P)\; Q$, then $R$ could have been of the form $(\lambda y.
                \lambda x. P') W$, hence of greater depth.

                \item Again, $R$ appears in $M$ as $(R\; Q)$ and after reduction
                becomes a new redex of the form $(\lambda x. P) Q$, then $R$
                could have also been of the form $(\lambda y. y) (\lambda x.
                P)$, which is again of greater depth.
            \end{enumerate}
    \end{enumerate}
    Hence in all cases either the value of $d_{max}$ is strictly smaller or the
    number of maximal redexes reduces and thus by induction hypothesis we are
    done.
\end{proof}

In fact, a much stronger result holds but we will not prove it here:
\begin{theorem}[Strong normalisation]
    Let $M \in \Lambda_\Pi$, then any sequence of $\beta$-reductions
    starting in $M$ terminates in a finite of number of steps in a normal form.
\end{theorem}
