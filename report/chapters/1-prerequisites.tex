\chapter{Pre-requisites}

We begin our journey to the correspondance by introducing all the required
background theory.

%% @todo: perhaps add an informal introduction of lambda-calculus here?

\section{Untyped $\lambda$-calculus}

%% Here I follow the style "Introduction to Lambda Calculus" by
%% H. Barendregt and E. Barendsen to avoid dealing with "pre-terms" and that
%% crap.

The language of untyped $\lambda$-calculus is surprisingly simple.

We begin with an infinite set of variables $V = \{v, v', v'', \ldots\}$ from
which we construct the set of $\lambda$-terms which we shall call $\Lambda$.

We define the set $\Lambda$ recursively by:
\begin{align*}
    x \in V & \Rightarrow x \in \Lambda &
        \text{variables} \\
    M \in \Lambda,\, x \in V & \Rightarrow (\lambda x M) \in \Lambda &
        \text{$\lambda$-abstraction} \\
    M, N \in \Lambda & \Rightarrow (M N) &
        \text{function application}
\end{align*}

\emph{Note}: We shall always use lower case letters to denote \emph{variables} and
uppercase letters to denote arbitrary $\lambda$-terms.

Intuitively, we think of $\lambda$-abstractions as functions, where the meaning
of the word 'function' lies somewhere inbetween our understanding of functions as
part of mathematics and functions as part of computer programs. We shall expand
on this idea in later parts.

In the same spirit, we think of $(M N)$ as ``function $M$ applied to $N$''.

To avoid clutter with excessive bracketing, we assume the usual conventions:
\begin{align*}
    \lambda x_1 x_2 \ldots x_n . M &\equiv
        (\lambda x_1 (\lambda x_2 (\cdots (\lambda x_n M)) \\
    F M N &\equiv ((FM)N)
\end{align*}

The $\lambda$ operator, in the same way as quantifiers in First-Order Logic,
binds the variable in front of it, hence we have the definition:
\begin{definition}[Free variables]
    The set of free variables of $M \in \Lambda$, denoted $FV(M)$ is defined
    recursively as:
        \begin{align*}
            FV(x) &= \{x\} \\
            FV(\lambda x M) &= FV(M) \setminus \{x\} \\
            FV(M N) &= FV(M) \cup FV(N)
        \end{align*}
\end{definition}

Conversly, we say that a variable $x$ occuring in $M$ is \emph{bound} if it is
not in $FV(M)$. We also say that $M$ is \emph{closed} iff $FV(M) = \emptyset$.

We denote $M[N/x]$ to mean substituting all free occurances of $x$ in $M$ by $N$.
%% Write it out explicitly here as in the book?

{\center \texttt{Need to talk about variable naming convention/avoiding free
    variable capture after substitution.}}

%% @todo: make this more formal, review if I have skipped too much.
\begin{definition}[$\alpha$-renaming]
Since our intuition is to think of $\lambda$-calculus as a way to express
computation, the variable names should not go in the way, therefore we identify
two $\lambda$-terms if they only differ in the names of bound variables, e.g.
$\lambda x x \equiv \lambda y y$. Furthermore, we can replace any subterm of a
$\lambda$-term by one which has been defined as equivalent.
\end{definition}

Now we move on to some results.

\subsection{$\beta$-reduction}

Since $\lambda$-calculus is supposed to express computable terms, we need to
have a way of evaluating them. This is achieved via something called
$\beta$-reduction. One can view $\beta$-reduction as yet another scheme for
replacements (such as $\alpha$-conversion).

\begin{definition}[$\beta$-reduction]
\[ (\lambda x M N) \rightarrow_\beta M[N/x] \]
\end{definition}

%% @todo: maybe this part is BS/not that much related to our paper, so I
%% don't know why I actually wrote this down
At this point we can see how functions (i.e. $\lambda$-abstractions) are
similar and differ from $\lambda$-calculus functions in mathematics:
Consider the identity function $f(x) = x$ and the equivalent $\lambda x. x$

An application of $f$ to $3$ in mathematics would correspond to: $f(3) = 3$.

Although $3$ is not an element of $\Lambda$, we can informally write:
    \[ (\lambda x. x\, 3) \rightarrow_\beta 3 \]

So in both cases what we did is just replacing the variable $x$ with $3$. Now
for the unsimilar part, since we can apply the $\lambda$-abstraction to
\emph{any} term, we can apply it to, say, another $\lambda$-abstraction:
    \[ (\lambda x. x\, \lambda y. 1) \rightarrow_\beta \lambda y. 1 \]

$\lambda y. 1$ can thought of as a constant function which always returns $1$.
In mathematics, this would correspond to something like: $g(x) = 1$.
The difference arrises when we try to apply $f$ on $g$. In mathematics, one
cannot write $f(g) = g$, we can only do explicit function composition $(f \circ
g)(x) = g(x) = 1$. This is because functions are not first class citizens in
the language. But fortunately they are in $\lambda$-calculus, which allows us
to do fun things.

Consider the abstraction $\lambda x.\; x\, x$, call it $\bm{D}$ for double.
What happens if we apply $\bm{D}$ to itself?
    \[ (\bm{D} \bm{D}) = (\lambda x.\, x\, x\: \bm{D})
        \rightarrow_\beta (\bm{D} \bm{D}) \]

So it seems that evaluating $(\bm{D} \bm{D})$ results in itself, which means
that if we were to write a program that would try to evaluate it, it would get
stuck in an infinite loop. We can even think of worse examples by taking
$\bm{T} = \lambda x.\; x\, x\,x$ and trying to evaluate $\bm{T}\, \bm{T}$.
Again, this hints on the similarity with computation (via non-terminating
programs).

\center{\texttt{Continue up to at least defining Church-Rosser}}

\section{Simply typed $\lambda$-calculus}

\section{Revisit of Natural Deduction}
