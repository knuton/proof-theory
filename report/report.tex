\documentclass[12pt,a4paper,titlepage]{scrreprt}
\usepackage{lmodern}
\usepackage[T1]{fontenc}
\usepackage[utf8]{inputenc}

\usepackage[natbib=true,style=authoryear-ibid]{biblatex}
\bibliography{references}
\usepackage{hyperref}
\hypersetup{hidelinks=true}

\usepackage{amsmath}
\usepackage{amsthm}
\usepackage{bm}
\usepackage{amssymb}
\usepackage{enumerate}

\usepackage{bussproofs}

\usepackage{microtype}
\setkomafont{sectioning}{\rmfamily\bfseries}

% "Macros" for re-occurring constructs
\newcommand{\limpl}{\mathcal{L}_\to}
\newcommand{\npiproofs}{\mathsf{D}_\to}
\newcommand{\implnpi}{$\to$\textbf{Npi}}
\newcommand{\implnpm}{$\to$\textbf{Npm}}
\newcommand{\stlambda}{$\lambda_\to$}
\newcommand{\stlambdaterms}{\Lambda_\Pi}
\newcommand{\chmap}{\mathcal{F}}
\newcommand{\chmapanta}{\mathcal{F}^{-1}}
\newcommand{\betared}{\rightarrow_\beta}

\theoremstyle{definition}
\newtheorem{definition}{Definition}[section]

\theoremstyle{theorem}
\newtheorem{theorem}{Theorem}[section]
\newtheorem{lemma}{Lemma}[section]
\newtheorem{proposition}{Proposition}[section]

\title{Curry-Howard correspondence for Ni and Church-Rosser theorem}
\author{%
    Ignas Vyšniauskas
        (\href{mailto:i.vysniauskas@gmail.com}{i.vysniauskas@gmail.com})\\
    Johannes Emerich
        (\href{mailto:Johannes@emerich.de}{johannes@emerich.de})
}

\begin{document}

\maketitle

\tableofcontents

% Preface
\chapter{Preface}

The two names usually connected with lambda calculus are those of Alonzo Church
and Haskell Curry. While both their work was groundlaying for the different
well-known systems of lambda calculus in use today, they originally set out to
design type-free formal systems that would allow capturing the foundations of
mathematics in a logical language with functional instead of set theoretic
notions. Both independently developed systems that were much larger than what is
known today as lambda calculus, or, in the case of Curry, combinatory logic.

It should turn out though, as was shown by Church's students Stephen Kleene and
John Rosser in 1934, that both systems were inconsistent as they allowed the
derivation of Richard's paradox. While this led Curry to attempt to correct his
approach, Church gave up on the foundational project and cocentrated his efforts
on those consistent parts of his system that had shown fruitful already in 1933
for making exact the notion of intuitively computable functions: the theory of
lambda terms. Along with the Turing Machine, this revelation should prove to be
foundational for the emerging field of computer science.

At around the same time in Europe, Andrei Kolmogorov from as early as 1925 and
Arendt Heyting from 1930 independently published articles in the seemingly
unrelated field of intuitionistic mathematics. In their work, they gave formal
systems for logic as it was interpreted in intuitionism, thereby making explicit
the interpretation of the logical connectives that is mainly due to Brouwer.
Brouwer saw as mathematical objects those entities that could be constructed in
the mind of the mathematician. This is in contrast to realist conceptions that
place mathematical objects outside of the mind, aswell as formalist conceptions
that cut ties of mathematics to any actual objects. While making mathematical
objects more malleable and accessible to the mathematician, intuitionism
precludes her from certain ways of reasoning about them. While for example the
realist has no trouble making a disjunctive statement about \textit{real}
objects, the intuitionist has nothing to make a disjunctive statement about if
she can not construct it in her mind, viz. show that at least one of the
disjuncts holds.

It should be clear that this conception of mathematics is incompatible with the
connectives of classical logic, as those would allow reasoning about
mathematical objects in the abstract which may lead to empty statements
possessing no referents. Instead, in order to make sensible propositions, the
mathematician needs to be able to construct what he is referring to. Naturally,
the notion of proofs is apt to change with the notions behind the connectives.
In this light, a step in a proof is valid if it is known how to constructively
prove the conclusion if its premises have been proven in the same manner. This
constructive perspective on proofs appears strikingly procedural and suggests
looking at proofs as objects that can be combined by rule application to form
new proofs. With some aquaintance with lambda calculus, and relatedly functional
programming, this understanding of proofs bears an uncanny resemblance to
different operations on functions. And indeed, in the years following the Second
World War, this connection has become more and more evident, being explicitly
shown for different proof systems and models of computation from the late 1950s.

The overall result is known as the \textit{Curry-Howard correspondence}. In this
report, we want to illuminate the correspondence between natural deduction for
propositional implicational logic and simply-typed lambda calculus. We will use
this result to look at parallels in normalisation of natural deduction proofs
and lambda terms, using the confluence (Church-Rosser) theorem to show the
consistency of \implnpi.


% Prerequisites
\chapter{Prerequisites}

% Just a first version, I think this is a good place to explain our "small"
% scope
We will give brief introductions to two prerequisites for the main part of the
report. First an introduction to simply typed lambda calculus, which we will
introduce first in an untyped version and add the notion of types on this basis.
Second, in \ref{sec:naturaldeduction}, a review of natural deduction for
intuitionistic propositional implication logic.  While results similar to those
presented in later chapters exist for larger logics, we chose this relatively
narrow scope because it allows us a concise presentation of both lambda calculus
and natural deduction.

%% @todo: perhaps add an informal introduction of lambda-calculus here?

\section{The Untyped $\lambda$-calculus}

The language of untyped $\lambda$-calculus is surprisingly simple.\footnote{Here we mostly follow the style used in \smartcite{introtolambda}.}

We begin with an infinite set of variables $V = \{v, v', v'', \ldots\}$ from
which we construct the set of $\lambda$-terms which we shall call $\Lambda$.

\begin{definition}[Untyped $\lambda$-calculus]\label{def:untyped-lambda-calc}
We define the set $\Lambda$ recursively by:
\begin{align*}
    x \in V & \Rightarrow x \in \Lambda &
        \text{variables} \\
    M \in \Lambda,\, x \in V & \Rightarrow (\lambda x M) \in \Lambda &
        \text{$\lambda$-abstraction} \\
    M, N \in \Lambda & \Rightarrow (M N) \in \Lambda &
        \text{function application}
\end{align*}
\end{definition}

\emph{Note}: We shall always use lower case letters to denote \emph{variables} and
uppercase letters to denote arbitrary $\lambda$-terms.

Intuitively, we think of $\lambda$-abstractions as functions, where the meaning
of the word `function' lies somewhere in between our understanding of functions as
part of mathematics and functions as part of computer programs. We shall expand
on this idea in later parts.

In the same spirit, we think of $(M N)$ as ``function $M$ applied to $N$''.

To avoid clutter with excessive bracketing, we assume the usual conventions:
\begin{align*}
    \lambda x_1 x_2 \ldots x_n . M &\equiv
        (\lambda x_1 (\lambda x_2 (\cdots (\lambda x_n M)) \\
    F M N &\equiv ((FM)N)
\end{align*}

The $\lambda$ operator, in the same way as quantifiers in First-Order Logic,
binds the variable in front of it, hence we have the definition:
\begin{definition}[Free variables]
    The set of free variables of $M \in \Lambda$, denoted $FV(M)$ is defined
    recursively as:
        \begin{align*}
            FV(x) &= \{x\} \\
            FV(\lambda x M) &= FV(M) \setminus \{x\} \\
            FV(M N) &= FV(M) \cup FV(N)
        \end{align*}
\end{definition}

Conversely, we say that a variable $x$ occurring in $M$ is \emph{bound} if it is
not in $FV(M)$. We also say that $M$ is \emph{closed} iff $FV(M) = \emptyset$.

%% God damn it, there is something wrong with this substitution stuff, but I
%% can't put my finger on it.
\begin{definition}[Substitution]
    Let $M,\, N \in \Lambda$, $x, y\in V$. We denote the
    substitution of (free occurances of) $x$ in $M$ by $N$ as $M[N/x]$ and
    define it as follows (here $x \neq y$):
    \begin{align*}
        x[N/x] &= N & \\
        y[N/x] &= y & \\
        (P Q)[N/x] &= P[N/x]\; Q[N/x] & \\
        (\lambda x. P)[N/x] &= \lambda x.P & \\
        (\lambda y. P)[N/x] &= \lambda y.P[N/x] & y \notin FV(N)
    \end{align*}
\end{definition}

%%% @todo: make this more formal, review if I have skipped too much.
%\begin{definition}[$\alpha$-renaming]
%Since our intuition is to think of $\lambda$-calculus as a way to express
%computation, the variable names should not go in the way, therefore we identify
%two $\lambda$-terms if they only differ in the names of bound variables, e.g.
%$\lambda x x \equiv \lambda y y$. Furthermore, we can replace any subterm of a
%$\lambda$-term by one which has been defined as equivalent.
%\end{definition}

Since $\lambda$-calculus is supposed to express computable terms, we need to
have a way of evaluating them. This is achieved via something called
$\beta$-reduction. One can view $\beta$-reduction as yet another scheme for
replacements (such as $\alpha$-conversion).

\begin{definition}[$\beta$-reduction] We define $\beta$-reduction as the
opperation $\betared$, such that:
\[ (\lambda x M) N \betared M[N/x] \]

It is important to note that $\betared$ induces a relation on $\Lambda$. We
want to be able to reduce subterms of any $\lambda$-term so we require that
$\betared$ is closed under the following rules:
\begin{align*}
    P \betared P' &\Rightarrow
        \begin{cases}
            \forall x \in V: & \lambda x.P \betared \lambda x. P' \\
            \forall Z \in \Lambda: & (P\, Z) \betared (P'\, Z) \\
            \forall Z \in \Lambda: & (Z\, P) \betared (Z\, P')
        \end{cases}
\end{align*}

\end{definition}

%% @todo: maybe this part is BS/not that much related to our paper, so I
%% don't know why I actually wrote this down
At this point we can see how functions (i.e. $\lambda$-abstractions) are
similar and differ from $\lambda$-calculus functions in mathematics:
Consider the identity function $f(x) = x$ and the equivalent $\lambda x. x$

An application of $f$ to $3$ in mathematics would correspond to: $f(3) = 3$.

Although $3$ is not an element of $\Lambda$, we can informally write:
    \[ (\lambda x. x\, 3) \betared 3 \]

So in both cases what we did is just replacing the variable $x$ with $3$. Now
for the unsimilar part, since we can apply the $\lambda$-abstraction to
\emph{any} term, we can apply it to, say, another $\lambda$-abstraction:
    \[ (\lambda x. x\, \lambda y. 1) \betared \lambda y. 1 \]

$\lambda y. 1$ can be thought of as a constant function which always returns
$1$.  In mathematics, this would correspond to something like: $g(x) = 1$.  The
difference arises when we try to apply $f$ on $g$. In mathematics, one cannot
write $f(g) = g$, we can only do explicit function composition $(f \circ g)(x) =
g(x) = 1$. This is because functions are not first class citizens in the
language. But fortunately they are in $\lambda$-calculus, which allows us to do
fun things.

Consider the abstraction $\lambda x.\; x\, x$, call it $\bm{D}$ for double.
What happens if we apply $\bm{D}$ to itself?
    \[ (\bm{D} \bm{D}) = (\lambda x.\; x\, x\: \bm{D})
        \betared (\bm{D} \bm{D}) \]

So it seems that evaluating $(\bm{D} \bm{D})$ results in itself, which means
that if we were to write a program that would try to evaluate it, it would get
stuck in an infinite loop. We can even think of worse examples by taking
$\bm{T} = \lambda x.\; x\, x\,x$ and trying to evaluate $\bm{T}\, \bm{T}$.
Again, this hints on the similarity with computation (via non-terminating
programs).

\begin{definition}[$\beta$-redex]
    A term of the form $(\lambda x. M) N$ is called a ($\beta$-)\emph{redex}.
\end{definition}
\begin{definition}[$\beta$-normal form]
    A term which can not be (further) $\beta$-reduced is said it to be in
    ($\beta$)-\emph{normal form}.
\end{definition}

Of a particular interest are iterated applications of $\betared$ and the
reflexive closure of the resulting relation:
\begin{definition}[$\betared^*$]
    $\betared^*$ is the transitive closure of $\betared$.
\end{definition}

\begin{definition}[$\leftrightarrow_\beta^*$]
    $\leftrightarrow_\beta^*$ is the reflexive closure of $\betared^*$.
\end{definition}

\begin{definition}[$=_\beta$]
    $=_\beta$ is the symmetric closure of $\leftrightarrow_\beta^*$, or in
    other words the equivalence relation induced by $\betared$.
\end{definition}

If for two $\lambda$-terms we have that $M =_\beta N$ then our intuition is
that they compute the same thing.

A crucial property of this relation is exposed by the following theorem.
\begin{theorem}[Church-Rosser Theorem]
    Let $M \in \Lambda$. If $M \betared^* M_1$ and $M \betared^* M_2$ then
    there exists $N$ s.t. $M_1 \betared^* N$ and $M_2 \betared^* N$.
\end{theorem}
\begin{proof}Omitted.\end{proof}

Here's a pretty diagram to illustrate the theorem:
\begin{diagram}
             &                 & M &                 &       \\
             & \ldTo_\beta     &   & \rdTo_\beta     &       \\
         M_1 &                 &   &                 &  M_2  \\
             & \rdDotsto_\beta &   & \ldDotsto_\beta &       \\
             &                 & N &                 &
\end{diagram}

\begin{corollary}[Uniqueness of normal form]
    If $M \betared^* N_1$ and $M \betared^* N_2$ and $N_1$ and $N_2$ are in
    normal form, then $N_1 = N_2$.
\end{corollary}

\begin{corollary}
    If $M_1 \betared^* N_1$ and $M_2 \betared^* N_2$, $N_1 \neq N_2$ and $N_1, N_2$
    are in normal form, then $M_1 \neq_\beta M_2$.
\end{corollary}


\section{The Natural Deduction System \implnpi}
\label{sec:naturaldeduction}

% I intend to add or redo passages as this becomes necessary for later sections

In this section we will briefly review natural deduction systems. In particular
we will look at the natural deduction system \implnpi\ for intuitionistic
propositional logic with implication as the only connective. We call the
corresponding language $\limpl$, its grammar is given by
$$\limpl ::=\ \perp |\ P\ |\ \limpl \to \limpl,$$
where $P$ is any of countably many propositional variable names.

We will conceptualize proofs as trees of formulas and proving as constructing
such trees. A formula $A$ is said to be \textit{deducible} in \implnpi\ if the
rules of the system allow the construction of a \textit{deduction tree} with its
root node being the conclusion $A$. Any non-leaf node $q$ possesses as its
children $n$ nodes that are the premisses of a rule application of which $q$ is
the conclusion. The leaf nodes of a deduction tree are its \textit{assumptions},
each labelled by a variable name as its marker. Multiple occurrences of the same
formula as assumptions with the same marker are said to compose an
\textit{assumption class}. If the marker labelling an assumption class is
mentioned in a rule application in the deduction tree, the assumption class is
called \textit{closed}, otherwise \textit{open}. The open assumptions in a
deduction tree are the assumptions of the deduction.

The rules for \implnpi\ allow as the simplest deduction of any $\limpl$ formula
$A$ the proof of $A$ from open assumption $A$ as the one-node deduction tree
consisting only of $A$. More complex deductions can be constructed by combining
any number of deduction trees by means of the system's only two rules, the
\textit{implication introduction rule} $\to$I and the
\textit{implication elimination rule} $\to$E:

\begin{center}
\AxiomC{$[A]^u$}
\noLine
\UnaryInfC{$\mathcal{D}_1$}
\noLine
\UnaryInfC{$B$}
\RightLabel{$\to$I, $u$}
\UnaryInfC{$A \to B$}
\DisplayProof
% TODO A bit more horizontal space would be nice
\AxiomC{$\mathcal{D}_1$}
\noLine
\UnaryInfC{$A \to B$}
\AxiomC{$\mathcal{D}_2$}
\noLine
\UnaryInfC{$A$}
\RightLabel{$\to$E}
\BinaryInfC{$B$}
\DisplayProof
\end{center}

In this notation, $\mathcal{D}_1$ and $\mathcal{D}_2$ are deduction trees. The
lines above and below a deduction tree are understood to be part of the tree
unless separated by a horizontal line. A horizontal line signifies a rule
application, with the name of the rule and optionally a variable name appearing
right of the line. $A$ and $B$ are formulas of $\mathcal{L}_\to$. $[A]^u$
denotes the assumption class of all occurrences of $A$ labelled with variable
name $u$. We see that $\to$I closes an assumption class, which we allow to
consist of zero occurrences of $A$.

% TODO Proper reference
This should clarify what we take ``\implnpi'' to mean, for details we refer to
Troelstra-Schwichtenberg whose general formalism we follow.



% Curry-Howard
\chapter{Curry-Howard Isomorphism}

Parts of the formalisms introduced in the prerequisites appear to function
largely in parallel. When we look at the effects of functional abstraction and
function application on types in \stlambda, we might be reminded of implication
introduction and implication elimination in \implnpi. And indeed, this apparent
structural similarity can be grasped formally and turns out to be so strong that
the two formalisms are isomorphic. To support this claim we obviously need to
establish the existence of a bijection between the two systems. We can then show
that the aforementioned operations are indeed invariant under the bijection.

In the literature we referred to, proofs of the isomorphism were largely
suggestive. Although we neither want to claim to give a totally complete proof,
nor want give a totally complete proof, we were interested in seeing the details
work out and thus try to take a more formal route compared to the sources.

\section{Smooth Operators}

\begin{definition}
We define the notation $M^{A}$ for a simply-typed lambda term $M$ to mean that
$M$ contains the outermost type annotation $A$, i.e. $M^A \equiv M':A$ for some
string $M'$.
\end{definition}

\begin{theorem}[Curry-Howard Isomorphism]

The natural deduction system \implnpi\ consisting of the set $\npiproofs$ of
deduction trees with implication introduction and implication elimination as
operations is isomorphic to the term calculus \stlambda\ consisting of the set
$\stlambdaterms$ of simply-typed lambda terms with functional abstraction and
function application as operations.

\end{theorem}

\begin{proof}
We give a bijection $\chmap: \npiproofs \to \stlambdaterms$ from the set $\npiproofs$
of \implnpi\ deduction trees to the set $\stlambdaterms$ of simply-typed lambda terms.

\begin{alignat*}{2}
\chmap(A^u) &:= u: A \\[6pt]
\chmap\left(
  \AxiomC{$\mathcal{D}_1$}
  \noLine
  \UnaryInfC{$A \to B$}
  \AxiomC{$\mathcal{D}_2$}
  \noLine
  \UnaryInfC{$A$}
  \RightLabel{$\to$E}
  \BinaryInfC{$B$}
  \DisplayProof
\right) &:= \chmap(\mathcal{D}_1) \chmap(\mathcal{D}_2): B \\[6pt]
\chmap\left(
  \AxiomC{$[A]^u$}
  \noLine
  \UnaryInfC{$\mathcal{D}$}
  \noLine
  \UnaryInfC{$B$}
  \RightLabel{$\to$I, $u$}
  \UnaryInfC{$A \to B$}
  \DisplayProof
\right) &:= (\lambda u: A.\chmap(\mathcal{D})): A \to B
\end{alignat*}

\begin{proposition}
If $\mathcal{D}$ is a deduction of $A$, then $\chmap(\mathcal{D})$ has type $A$.
\end{proposition}

\begin{proof}
By induction on the complexity of $\mathcal{D}$.
\end{proof}

We additionally define a function $\stlambdaterms \to \npiproofs$, suggestively
named $\chmapanta$:

\begin{alignat*}{2}
\chmapanta(u: A) & := A^u \\[6pt]
\chmapanta((M^{A \to B})(N^A): B) &:=
  \AxiomC{$\chmapanta(M^{A \to B})$}
  \noLine
  \UnaryInfC{$A \to B$}
  \AxiomC{$\chmapanta(N^A)$}
  \noLine
  \UnaryInfC{$A$}
  \RightLabel{$\to$E}
  \BinaryInfC{B}
  \DisplayProof \\[6pt]
\chmapanta((\lambda u:A.M^B):A \to B) &:=
  \AxiomC{$\chmapanta(M^B)$}
  \noLine
  \UnaryInfC{$B$}
  \RightLabel{$\to$I, $u$}
  \UnaryInfC{$A \to B$}
  \DisplayProof
\end{alignat*}

\begin{proposition}
For any lambda term $M^A$, $\chmapanta(M^A)$ is a proof of $A$.
\end{proposition}

\begin{proof}
By induction on the complexity of lambda terms.
\end{proof}

We now show that $\chmap$ is a bijection by showing that
\begin{itemize}
\item[(i)] $\chmapanta \circ \chmap = id_{\npiproofs}$ and
\item[(ii)] $\chmap \circ \chmapanta = id_{\stlambdaterms}$.
\end{itemize}
The proof is by induction on the construction of deduction trees and lambda
terms respectively.

\[
\chmapanta(\chmap(A^u)) = \chmapanta(u:A) = A^u = id_{\npiproofs}(A^u)
\]

\begin{align*}
\chmapanta\left(\chmap\left(
  \AxiomC{$\mathcal{D}_1$}
  \noLine
  \UnaryInfC{$A \to B$}
  \AxiomC{$\mathcal{D}_2$}
  \noLine
  \UnaryInfC{$A$}
  \RightLabel{$\to$E}
  \BinaryInfC{$B$}
  \DisplayProof
\right)\right)
& =
  \chmapanta(\chmap(\mathcal{D}_1) \chmap(\mathcal{D}_2): B)
  \tag{By def. $\chmap$} \\[6pt]
& =
  \AxiomC{$\chmapanta(\chmap(\mathcal{D}_1))$}
  \noLine
  \UnaryInfC{$A \to B$}
  \AxiomC{$\chmapanta(\chmap(\mathcal{D}_2))$}
  \noLine
  \UnaryInfC{$A$}
  \RightLabel{$\to$E}
  \BinaryInfC{B}
  \DisplayProof \tag{By def. $\chmapanta$} \\[6pt]
& =
  \AxiomC{$\mathcal{D}_1$}
  \noLine
  \UnaryInfC{$A \to B$}
  \AxiomC{$\mathcal{D}_2$}
  \noLine
  \UnaryInfC{$A$}
  \RightLabel{$\to$E}
  \BinaryInfC{$B$}
  \DisplayProof \tag{By induction hypothesis} \\[6pt]
& =
  id_{\npiproofs}\left(
  \AxiomC{$\mathcal{D}_1$}
  \noLine
  \UnaryInfC{$A \to B$}
  \AxiomC{$\mathcal{D}_2$}
  \noLine
  \UnaryInfC{$A$}
  \RightLabel{$\to$E}
  \BinaryInfC{$B$}
  \DisplayProof
  \right)
\end{align*}

\begin{align*}
\chmapanta\left(\chmap\left(
  \AxiomC{$[A]^u$}
  \noLine
  \UnaryInfC{$\mathcal{D}$}
  \noLine
  \UnaryInfC{$B$}
  \RightLabel{$\to$I, $u$}
  \UnaryInfC{$A \to B$}
  \DisplayProof
\right)\right)
& =
  \chmapanta((\lambda u: A.\chmap(\mathcal{D})): A \to B)
  \tag{By def. $\chmap$} \\[6pt]
& =
  \AxiomC{$\chmapanta(\chmap(\mathcal{D}))$}
  \noLine
  \UnaryInfC{$B$}
  \RightLabel{$\to$I, $u$}
  \UnaryInfC{$A \to B$}
  \DisplayProof \tag{By def. $\chmapanta$} \\[6pt]
& =
  \AxiomC{$[A]^u$}
  \noLine
  \UnaryInfC{$\mathcal{D}$}
  \noLine
  \UnaryInfC{$B$}
  \RightLabel{$\to$I, $u$}
  \UnaryInfC{$A \to B$}
  \DisplayProof \tag{By induction hypothesis} \\[6pt]
& =
  id_{\npiproofs}\left(
  \AxiomC{$[A]^u$}
  \noLine
  \UnaryInfC{$\mathcal{D}$}
  \noLine
  \UnaryInfC{$B$}
  \RightLabel{$\to$I, $u$}
  \UnaryInfC{$A \to B$}
  \DisplayProof
  \right)
\end{align*}

\begin{align*}
\chmap(\chmapanta(u:A)) & = \chmap(A^u) = u:A = id_{\stlambdaterms}(u:A)
\end{align*}

\begin{align*}
\chmap(\chmapanta(M^{A \to B}N^A:B)) & =
  \chmap\left(
  \AxiomC{$\chmapanta(M^{A \to B})$}
  \noLine
  \UnaryInfC{$A \to B$}
  \AxiomC{$\chmapanta(N^A)$}
  \noLine
  \UnaryInfC{$A$}
  \RightLabel{$\to$E}
  \BinaryInfC{B}
  \DisplayProof
  \right) \tag{By def. $\chmapanta$}\\[4pt]
& =
  \chmap(\chmapanta(M^{A \to B}))\chmap(\chmapanta(N^A)): B
  \tag{By def.  $\chmap$}\\[6pt]
& =
  M^{A \to B}N^A:B
  \tag{By induction hypothesis}\\[6pt]
& =
  id_{\stlambdaterms}(M^{A \to B}N^A:B)
\end{align*}

\begin{align*}
\chmap(\chmapanta((\lambda u:A.M^B):A \to B)) &=
  \chmap\left(
  \AxiomC{$\chmapanta(M^B)$}
  \noLine
  \UnaryInfC{$B$}
  \RightLabel{$\to$I, $u$}
  \UnaryInfC{$A \to B$}
  \DisplayProof
  \right) \tag{By def. $\chmapanta$} \\[6pt]
& =
  (\lambda u: A.\chmap(\chmapanta(M^B))): A \to B
  \tag{By def. $\chmap$} \\[6pt]
& =
  (\lambda u: A.M^B): A \to B
  \tag{By induction hypothesis} \\[6pt]
& =
  id_{\stlambdaterms}((\lambda u: A.M^B): A \to B)
\end{align*}

The correspondence of implication introduction with functional abstraction, and
implication elimination with function application is immediate from the
definition of $\chmap$ and $\chmapanta$.
\end{proof}

\section{Combinators and Axioms}
\label{section:ski}

The main result of the previous section is known under various names, among them
the \textit{formulas-as-types} and \textit{proofs-as-programs} interpretations.
When we showed three exemplary deductions of axioms of propositional logic in
\ref{sec:naturaldeduction}, we took care to sneak in something that would make
sense in the light of these interpretations. The three axioms that were deduced
correspond to typed instances of three \textit{combinators} that are known as
$\mathsf{I}$, $\mathsf{K}$ and $\mathsf{S}$ in the literature:
\begin{align*}
\mathsf{I} &=
  (\lambda u^A.u)^A \\
\mathsf{K} &=
  (\lambda u^A.
     (\lambda v^B.u)
  )^{A \to B \to A} \\
\mathsf{S} &=
  (\lambda u^{A \to B \to C}.
     (\lambda w^{A \to B}.
       (\lambda v^A.(uv)(wv))
     )
  )^{(A \to B \to C) \to ((A \to B) \to (A \to C))}
\end{align*}
The system consisting of the $\mathsf{SKI}$ combinators is a well-studied Turing
complete system. While $\mathsf{SKI}$ due to its bare manner is of theoretical
interest only, it is not a minimal system. Instead, already the system
$\mathsf{SK}$ is Turing complete, as $\mathsf{I}$ can be obtained by combining
$\mathsf{S}$ and $\mathsf{K}$. Looking at the combinators as given above, we see
a problem with this. We can neither apply $\mathsf{S}$ to $\mathsf{K}$ nor
$\mathsf{K}$ to $\mathsf{S}$, because the types don't match.

Unless we treat each of the combinators as a schema to be instantiated by
assigning types to the type meta-variables occuring in the expression, we will
have to construct appropriate combinators for specific types. Because of the
workings of \implnpi\ we know that we could always do that by using the
appropriate simple types from $\Pi$ as assumptions in the appropriate places. We
can therefore justify the meta-rule of using the combinators above as schemas.

We will now combine appropriate instances of $\mathsf{S}$ and $\mathsf{K}$ to
yield $\mathsf{I}$. To better manage the size of the expressions, we will omit
types where unproblematic and use variables for terms.
\begin{align*}
\mathsf{S} &=
(\lambda u^{(A \to (B \to A) \to A) \to (A \to (B \to A)) \to (A \to A)}.
  (\lambda w^{A \to B \to A}.
    (\lambda v^A.(uv)(wv))
  )
) \\
\mathsf{K_1} &=
(\lambda u_1^A.
  (\lambda v_1^{B \to A}.u)
) \\
\mathsf{K_2} &=
(\lambda u_2^A.
  (\lambda v_2^B.u)
) \\
\mathsf{(SK_1)K_2} &=_\beta
(
  (\lambda u^{(A \to (B \to A) \to A) \to (A \to (B \to A)) \to (A \to A)}.
    (\lambda w^{A \to B \to A}.
      (\lambda v^A.(uv)(wv))
    )
  )
  \mathsf{K_1}
)
\mathsf{K_2} \\
&=_\beta
(\lambda w^{A \to B \to A}.
  (\lambda v^A.(\mathsf{K_1}v)(wv))
)
\mathsf{K_2} \\
&=_\beta
(\lambda v^A.(\mathsf{K_1}v)(\mathsf{K_2}v)) \\
&=_\beta
(\lambda v^A.
  ((\lambda u_1^A.
    (\lambda v_1^{B \to A}.u)
  )v)
  (\mathsf{K_2}v)
) \\
&=_\beta
(\lambda v^A.
  (\lambda v_1^{B \to A}.v)
  (\mathsf{K_2}v)
) \\
&=_\beta
(\lambda v^A.
  (\lambda v_1^{B \to A}.v)
  ((\lambda u_2^A.
     (\lambda v_2^B.u_2)
  )v)
) \\
&=_\beta
(\lambda v^A.
  (\lambda v_1^{B \to A}.v)
  (\lambda v_2^B.v)
) \\
&=_\beta
(\lambda v^A.v) \\
&=
\mathsf{I}
\end{align*}
We were thus able to derive $\mathsf{I}$, given we identify terms that differ only in
the names of bound variables.

It would be interesting to see what transformations are being made on the
corresponding deduction trees while following steps of $\beta$-reduction.  Alas,
the terms above become very complicated when translated into deductions. We will
have to look at a simpler example.

\section{Normalisation}

\begin{definition}[Proof Instantiation]
We define a function for replacing all occurences of an open assumption $A^u$ in
a deduction tree $\mathcal{D}$ by a deduction tree $\mathcal{E}$, written as
$\mathcal{D}[A^u/\mathcal{E}]$.

\[
  \mathcal{D}[A^u/\mathcal{E}] =
  \left\{
  \begin{array}{ll}
  \mathcal{E} & \mbox{if } \mathcal{D} = A^u \\[6pt]
  \mathcal{D} & \mbox{if } \mathcal{D} = B^v, v \neq u \\[10pt]
  \mathcal{D} & \mbox{if } \mathcal{D} =
    \AxiomC{$\mathcal{D}'$}
    \noLine
    \UnaryInfC{$C$}
    \RightLabel{$\to$I, $u$}
    \UnaryInfC{$B \to C$}
    \DisplayProof \\[20pt]
  \AxiomC{$\mathcal{D}'[A^u/\mathcal{E}]$}
  \noLine
  \UnaryInfC{$C$}
  \RightLabel{$\to$I, $v$}
  \UnaryInfC{$B \to C$}
  \DisplayProof & \mbox{if } \mathcal{D} =
    \AxiomC{$\mathcal{D}'$}
    \noLine
    \UnaryInfC{$C$}
    \RightLabel{$\to$I, $v$}
    \UnaryInfC{$B \to C$}
    \DisplayProof, v \neq u \\[24pt]
  \AxiomC{$\mathcal{D}'[A^u/\mathcal{E}]$}
  \noLine
  \UnaryInfC{$B \to C$}
  \AxiomC{$\mathcal{D}''[A^u/\mathcal{E}]$}
  \noLine
  \UnaryInfC{$B$}
  \RightLabel{$\to$E}
  \BinaryInfC{$C$}
  \DisplayProof & \mbox{if } \mathcal{D} =
    \AxiomC{$\mathcal{D}'$}
    \noLine
    \UnaryInfC{$B \to C$}
    \AxiomC{$\mathcal{D}''$}
    \noLine
    \UnaryInfC{$B$}
    \RightLabel{$\to$E}
    \BinaryInfC{$C$}
    \DisplayProof
  \end{array}
  \right.
\]
\end{definition}

\begin{definition}[Conversion]
We define a relation $\leadsto \subseteq \npiproofs\times\npiproofs$ that captures the notion of
normalisation steps for \implnpi\ deductions.

\[
  \AxiomC{$[A]^u$}
  \noLine
  \UnaryInfC{$\mathcal{D}$}
  \noLine
  \UnaryInfC{$B$}
  \RightLabel{$\to$I, $u$}
  \UnaryInfC{$A \to B$}
  \AxiomC{$\mathcal{D}_1$}
  \noLine
  \UnaryInfC{$A$}
  \RightLabel{$\to$E}
  \BinaryInfC{$B$}
  \DisplayProof
\ \leadsto\ \mathcal{D}[A^u/\mathcal{D}_1]
\]

\[
  \AxiomC{$[A]^u$}
  \noLine
  \UnaryInfC{$\mathcal{D}$}
  \noLine
  \UnaryInfC{$B$}
  \RightLabel{$\to$I, $u$}
  \UnaryInfC{$A \to B$}
  \DisplayProof
\ \leadsto\ 
  \AxiomC{$[A]^u$}
  \noLine
  \UnaryInfC{$\mathcal{E}$}
  \noLine
  \UnaryInfC{$B$}
  \RightLabel{$\to$I, $u$}
  \UnaryInfC{$A \to B$}
  \DisplayProof
\mbox{if } \mathcal{D} \leadsto \mathcal{E}
\]

\[
  \AxiomC{$\mathcal{D}_1$}
  \noLine
  \UnaryInfC{$A \to B$}
  \AxiomC{$\mathcal{D}_2$}
  \noLine
  \UnaryInfC{$A$}
  \RightLabel{$\to$E}
  \BinaryInfC{$B$}
  \DisplayProof
\ \leadsto\ 
  \AxiomC{$\mathcal{E}_1$}
  \noLine
  \UnaryInfC{$A \to B$}
  \AxiomC{$\mathcal{E}_2$}
  \noLine
  \UnaryInfC{$A$}
  \RightLabel{$\to$E}
  \BinaryInfC{$B$}
  \DisplayProof
\mbox{ if } \mathcal{D}_1 \leadsto \mathcal{E}_1
\mbox{ or } \mathcal{D}_1 \leadsto \mathcal{E}_1
\]
\end{definition}



\printbibliography

\end{document}
